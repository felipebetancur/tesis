\newpage
\begin{abstract}
	
	\begin{normalsize}

	En esta tesis se presenta un análisis del impacto producido en la performance de clasificación al entrenar un clasificador de caracteres con imágenes sintéticas (Wang et al., 2011). El objetivo es clasificar caracteres en escenas naturales en donde las técnicas tradicinales de OCR no se pueden aplicar de forma directa (De Campos et al., 2009). Para realizar esto, se modifican imágenes de fuentes a través de diferentes transformaciones con el objetivo de simular las condiciones de las imágenes reales. Se complementa este análisis realizando una análisis de performance utilizando diferentes conjuntos de entrenamiento sintéticos generados a partir del dataset público conocido como \textit{Chars74k}. El resultado final de este trabajo sirve para corrobar que la generación de datos sintéticos produce un impacto positivo en la clasificación y más aún si estos se combinan con datos reales.

		{\bf Keywords:} computer vision, Random Ferns, classification, supervised algorithm.

	\end{normalsize}

\end{abstract}