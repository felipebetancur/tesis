\newpage
\begin{abstract}
	
	\begin{normalsize}

	El reconocimiento de texto siempre ha sido un desafio a nivel computacional. Partiendo del reconocimiento de textos mecanografiados, la dificultad se acentúa enormemente cuando se intenta lograr lo mismo con textos presentes en condiciones naturales como la letra manuscrita o el texto que se puede encontrar en la etiqueta de algún producto. El que una computadora pueda discernir un carácter de otro en la imagen de un texto no es una tarea sencilla. El objetivo es clasificar caracteres en escenas naturales en donde las técnicas tradicinales de OCR no se pueden aplicar de forma directa (De Campos et al., 2009). En esta tesis se presenta un análisis del impacto producido en la performance de clasificación al entrenar un clasificador de caracteres con imágenes sintéticas (Wang et al., 2011). Se complementa esto realizando una análisis de performance utilizando diferentes conjuntos de entrenamiento sintéticos generados a partir del dataset público conocido como \textit{Chars74k}. El resultado final de este trabajo sirve para corrobar que este tipo de datos produce un impacto positivo en la clasificación y más aún si estos se combinan con datos reales.

		{\bf Keywords:} computer vision, Random Ferns, classification, supervised algorithm.

	\end{normalsize}

\end{abstract}