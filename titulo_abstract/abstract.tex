\newpage
\begin{abstract}
	
	\begin{normalsize}

	En esta tesis, se presenta un análisis del impacto producido en la performance de clasificación al entrenar un clasificador de caracteres con imágenes sintéticas (Wang et al., 2011). El objetivo, es clasificar caracteres en imágenes naturales por lo cual las técnicas tradicinales de OCR no se pueden aplicar de forma directa (De Campos et al., 2009). Para realizar esto, se modifican imágenes de fuentes a través de diferentes transformaciones afines con el objetivo de simular las condiciones de las imágenes naturales. Se complementa este análisis realizando una comparación de performance utilizando diferentes tipos de datasets, como así también comparando los resultados con los obtenidos en condiciones similar por Wang et al. El resultado final de este trabajo sirve para ....\textbf{continuará cuando termine el borrador final}.

		{\bf Keywords:} computer vision, Random Ferns, classification, pattern recognition, supervised algorithm.

	\end{normalsize}

\end{abstract}