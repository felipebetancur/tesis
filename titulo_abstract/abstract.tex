\newpage
\begin{abstract}
	
	\begin{normalsize}

	En esta tesis, se presenta un análisis del impacto producido en la performance de clasificación al entrenar un clasificador de caracteres con imágenes sintéticas (Wang et al., 2011). El objetivo, es clasificar caracteres en escenas naturales por lo cual las técnicas tradicinales de OCR no se pueden aplicar de forma directa (De Campos et al., 2009). Para realizar esto, se modifican imágenes de fuentes a través de diferentes transformaciones afines con el objetivo de simular las condiciones de las imágenes reales. Se complementa este análisis realizando una análisis de performance utilizando diferentes conjuntos de entrenamiento sintéticos generados a partir de un dataset público reconocido \textit{Chars74k}, como así también comparando los resultados con los obtenidos en condiciones similares por Wang et al. Esto elimina la necesidad de recolectar y etiquetar datos de entrenamiento del mundo real que de otro modo conlleva a un gasto de tiempo importante. El resultado final de este trabajo sirve para corrobar no sólo que...existen formas de resolver el problema del reconocimiento de texto en escenas naturales....que la generación de datos sintéticos genera un impacto positivo en la clasificación... dejar ver las dificultades inherentes al reconocimiento de texto en imágenes naturales y que actualmente sigue sin haber un método eficiente (similar al obtenido con los documentos escaneados) para lograr un reconocimiento eficaz.

		{\bf Keywords:} computer vision, Random Ferns, classification, pattern recognition, supervised algorithm.

	\end{normalsize}

\end{abstract}