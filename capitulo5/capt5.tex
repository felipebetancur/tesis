\newpage
\section{Conclusiones y trabajos futuros}

	Para concluir con este trabajo y en base a los resultados expuestos, se puede concluir que la tarea de reconocimiento de caracteres en imágenes estructuradas no es una tarea sencilla. Actualmente han habido bastantes avances en el campo pero todavía no  se logra desarrollar un método que compita con los clasificadores de documentos escaneados.
	
	Basados en los análisis del capítulo 4, se pudo observar que el uso de imágenes sintéticas proporcionó mejoras al momento de entrenar al clasificador si se las combina con imágenes reales. Si bien la mejora no es muy notoria, se puede seguir trabajando para mejorar esto y así poder evitar a futuro el tedioso trabajo de tener que recolectar imágenes. El uso de Random Ferns como clasificador es una buena opción por lo expuesto en el capítulo 2; sin embargo, sería interesante probar con otros para poder realizar una mejor comparación.
	
	Sin duda alguna el desafio de reconocer texto en escenas naturales es un problema que vale la pena estudiar y pulir ya que las aplicaciones que tiene actualmente son enormes. Más aún con la proliferación de los dispositivos móviles y la necesidad de crear aplicaciones que asistan a usuarios con deficiencia visual entre otros.