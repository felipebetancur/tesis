	\subsection{Algunas definiciones en torno a la teledetecci\'on}
		El concepto de teledetecci\'on ha sido redefinido muchas veces a lo largo del tiempo:
		\begin{itemize}
			\item Seg\'un Fischer et al. (1986 p. 34), la teledetecci\'on es el arte o ciencia de decir algo sobre un objecto pero sin tocarlo.
			\item Seg\'un Lintz y Simonett (1976, p. 1), la teledetecci\'on es la adquisici\'on f\'isica de datos sobre un objeto sin tocarlo ni tener contacto con \'este.
			\item Seg\'un la Sociedad Americana de Fotograf\'ia, las im\'agenes son adquiridas utilizando un sensor, a diferencia de otros m\'etodos como el uso de las c\'amaras convencionales. El mismo, graba la escena utilizando escaneo electr\'onico, pudiendo tambi\'en utilizar radiaci\'on fuera del rango visual normal de las c\'amaras o las filmaciones. Ejemplos de \'esto, son las microondas, radares, im\'agenes t\'ermicas, infrarrojas, ultravioletas, as\'i como tambi\'en, multiespectrales. T\'ecnicas especiales son aplicadas para procesar e interpretar las im\'agenes de teledetecci\'on, con el fin de aportar datos a distintos sectores como agricultura, arqueolog\'ia, informaci\'on forestal, geograf\'ia, geolog\'ia y otras \'areas.
			\item La teledetecci\'on es la observaci\'on de un objetivo utilizando alg\'un dispositivo alejado por alguna distancia (Barrett y Curtis, 1976 p. 3)
			\item El t\'ermino ``teledetecci\'on'' en un sentido m\'as amplio, significa “reconocimiento a distancia” ( Colwell, 1966, p. 71)
			\item La teledetecci\'on, a pesar de no estar precisamente definida, incluye todos los m\'etodos para obtener im\'agenes u otras formas de grabaciones electromagn\'eticas de la superficie terrestre, desde la distancia, junto con el tratamiento y procesamiento de estos datos. Luego, la teledetecci\'on en el sentido amplio,  se refiere a la detecci\'on y grabaci\'on de radiaci\'on electromagn\'etica desde el punto de vista del instrumento sensor. Esta radiaci\'on pudo haber sido originada directamente desde el sensor o desde una fuente ajena, como la energ\'ia solar reflejada. (White, 1977, p. 1–2).
			\item Teledetecci\'on, es un t\'ermino usado por muchos cient\'ificos para denominar el estudio de objetos remotos desde grandes distancias (superficies como la de la Tierra, la Luna, la Atm\'ostfera, incluso fen\'omenos estelares y gal\'acticos). En t\'erminos generales, la teledetecci\'on denota la utilizaci\'on de modernos sensores, equipamento de procesamiento de datos, teor\'ia de la informaci\'on y metodolog\'ia de procesamiento, teor\'ia de la comunicaci\'on y dispositivos, veh\'iculos espaciales y terrestres, sumados a sistemas te\'oricos y pr\'acticos, con el prop\'osito de llevar a cabo reconocimientos a\'ereos o en el espacio de la superficie de la tierra (National Academy of Sciences, 1970, p. 1).
			\item La teledetecci\'on es la ciencia de derivar informaci\'on de un objeto utilizando mediciones hechas a distancia y sin tomar contacto con \'el. La medida comunmente usada para sistemas de teledetecci\'on, es la energ\'ia electromagn\'etica que emana el objeto de inter\'es, aunque existen otras posibilidades como ondas s\'ismicas, ondas sonoras y tambi\'en la fuerza gravitacional (D. A. Landgrebe, quoted in Swain and Davis, 1978, p. 1).
			\item La teledetecci\'on es la pr\'actica de obtener informaci\'on de la superficie de la Tierra usando im\'agenes adquiridas desde una perspectiva a\'erea utilizando para ello radiaci\'on electromagn\'etica reflejada o emitida desde la superficie misma en una o m\'as regiones del espectro electromagn\'etico (Introduction to Remote Rensing, Fifth Edition 2011, Campbell, Wynne).
		\end{itemize}