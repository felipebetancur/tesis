\subsection{Trabajos relacionados}
	\textbf{Falta trabajar más}
	Otro trabajo que afrontó el problema de la detección de texto en imágenes naturales es el realizado por Campos et al. en \cite{dCBV09} donde comparan la performance de varios clasificadores(dentro de los cuales hay un motor de OCR comercial conocido como ABBYY FineReader) sobre un dataset que ellos mismos crearon llamado \textit{Char74K} y es el dataset sobre el cual corren varios experimentos de este trabajo, que se van a mostrar en capítulos posteriores. Las conclusiones del trabajo son interesantes ya que dentro de las mismas destacan la dificultad que tienen los motores de OCR al momento de clasificar caracteres en imágenes naturales.
	
	K. Wang y S. Belongie proponen en ~\cite{WangBelongie} un nuevo enfoque en el reconocimiento de palabras, basado en metodos basados en el reconocimiento de objetos genéricos. Tratan a cada palabra en el lexicón como una categoría de objeto y realizan un reconocimiento de categoría de palabra. Este trabajo tiene muchas similitudes con el presente ya que fue usado como base para realizar ~\cite{wang}.
	
	B. Gatos et al. en \cite{GPP03} proponen una nueva metodología que ayuda a la detección, la segmentación y el reconocimiento automático de texto en imágenes naturales. Básicamente, la metodología consiste en lograr una eficiente binarización de las imágenes naturales. Posteriormente se procesan las imagenes binarizadas para extraer el texto y reconocerlo mediante el uso de un motor de OCR.