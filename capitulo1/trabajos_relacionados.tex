\subsection{Trabajos relacionados}

	\RC{Por el momento sólo agrego trabajos relacionados sin estructurar bien la sección}
	
	Se han propuesto muchos enfoques para afrontar el problema del reconocimiento de texto en imágenes naturales. De Campos et al. en \cite{dCBV09} comparan la performance de varios clasificadores (dentro de los cuales hay un motor de OCR comercial conocido como \textit{ABBYY FineReader}) sobre un dataset que ellos mismos crearon llamado \textit{Chars74K} y es el dataset sobre el cual corren varios experimentos de este trabajo. Las conclusiones del trabajo son interesantes, ya que dentro de las mismas destacan la dificultad que tienen los motores de OCR al momento de clasificar caracteres en imágenes naturales. Además, remarcan los beneficios de usar datos sintéticos como fuentes para el entrenamiento los cuales logran un porcentaje de reconocimiento muy similar al obtenido con imagenes reales. Por último concluyen que sigue siendo una tarea difícil la del reconocimiento de caracteres manuscritos.
	
	Otro enfoque interesante lo proponen B. Gatos et al. en \cite{GPP03}. El mismo, consiste en una nueva metodología que ayuda a la detección, la segmentación y el reconocimiento automático de texto en imágenes naturales. Básicamente, la metodología consiste en lograr una eficiente binarización de las imágenes naturales. Posteriormente se procesan las imagenes binarizadas para extraer el texto y reconocerlo mediante el uso de un motor de OCR.

	L. Neumann y J. Matas se diferencian de los enfoques tradicionales que constan de pipelines de procesamiento y lo reemplaza con un marco de trabajo que consta en la verificación de hipótesis procesando de manera simultaneas múltiples lineas de texto. Además usan fuentes sintéticas como conjunto de entrenamiento, pero una particularidad de este enfoque es que a ninguna de estas fuentes se les aplican transformaciones afines o rotaciones. Los resultados obtenidos son bastente prometedores.
	
		K. Wang y S. Belongie proponen en ~\cite{WangBelongie} un nuevo enfoque en el reconocimiento de palabras, asentado en metodos basados en el reconocimiento de objetos genéricos. Tratan a cada palabra en el lexicón como una categoría de objeto y realizan un reconocimiento de categoría de palabra. Este trabajo tiene muchas similitudes con el presente ya que fue usado como base para realizar ~\cite{wang}.