\subsection{Estructura de la tesis}

	La tesis se desarrolla a lo largo de 5 capítulos.	
		
	En el capítulo 1 se procede a explicar los conceptos teóricos que involucran el presente trabajo. Principalmente se abordan los conceptos básicos de aprendizaje supervisado con una ligera introducción a conceptos básicos de probabilidad que son de utilidad a lo largo de todo el capítulo. Posteriormente se describen los conceptos de clasificación necesarios para poder introducir a nuestro clasificador. Finalmente, se introducen los conceptos de imágenes como sus generalidades, características locales, entre otros.
	
	 En el capítulo 3, se describe el trabajo realizado por Wang et al. en \cite{wang} y se explica que partes del mismo se han implementado en la presente tesis.

	En el capítulo 4, se abordan los experimentos realizados en el trabajo. Se describe tanto la implemetación del pipeline de procesamiento, como así también su diseño, el dataset usado y los resultados obtenidos. Por último se realiza un análisis completo de los resultados.
	
	En el último capítulo, las conclusiones y trabajos futuros, se hace un resumen de lo logrado a lo largo de esta tesis así como un descripción de los objetivos futuros que se pretenden seguir en este trabajo.
	
	\textbf{Falta completarlo más - Rodri}