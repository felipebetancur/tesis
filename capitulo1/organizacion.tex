\subsection{Estructura de la Tesis}

	Esta tesis se desarrolla a lo largo de 5 capítulos.	
		
	En el capítulo 2 se procede a explicar los conceptos teóricos que involucran el presente trabajo. Principalmente se abordan los principios básicos del aprendizaje supervisado. Posteriormente se describen los conceptos necesarios para poder introducir  el clasificador Random Ferns. Finalmente, se realiza una introducción a las nociones básicas del procesamiento de imágenes.
	
	 En el capítulo 3, se describe el trabajo realizado por Wang et al. en \cite{wang} y se explica qué partes del mismo se han implementado en la presente tesis.

	En el capítulo 4, se abordan los experimentos realizados en el trabajo. Se describe tanto la implemetación del pipeline de procesamiento, como así también su diseño, el dataset usado y los resultados obtenidos. Por último se realiza un análisis completo de los resultados.
	
	En el último capítulo, de conclusiones y trabajos futuros, se hace un resumen de lo logrado a lo largo de esta tesis así como un descripción de los objetivos futuros que se pretenden seguir en este trabajo.
