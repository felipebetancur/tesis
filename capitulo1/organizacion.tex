\subsection{Estructura de la tesis}

	Se desarrollará la tesis a lo largo de 5 capítulos.
	
	En el próximo capítulo se procederán a explicar los conceptos teóricos que involucran el presente trabajo. Principalmente se abordarán los conceptos básicos de aprendizaje supervisado con una ligera introducción a conceptos básicos de probabilidad que serán de utilidad a lo largo de todo el capítulo. Posteriormente se describirán los conceptos de clasificación donde se describirán todos los conceptos necesarios para poder introducir a nuestro clasificador. Finalmente, se introducirán los conceptos de imágenes como sus generalidades, características locales, entre otros.
	
	 En el capítulo 3, se describirá el trabajo realizado por Wang et al. en \cite{wang} y se explicará que partes del mismo se han implementado en la presente tesis.

	En el capítulo 4, se abordan los experimentos realizados en el trabajo. Se describe tanto la implemetación del pipeline de procesamiento, como así también su diseño, el dataset usado y los resultados obtenidos. Por último se realiza un análisis completo de los resultados.
	
	En el último capítulo, las conclusiones y trabajos futuros, se hace un resumen de lo logrado a lo largo de esta tesis así como un descripción de los objetivos futuros que se pretenden seguir en este trabajo.
	
	\textbf{Falta completarlo más - Rodri}