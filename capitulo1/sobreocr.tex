\subsection{Sobre Reconocimiento óptico de caracteres}

	\subsubsection{Definición del problema}
	
	El reconocimiento óptico de caracteres, usualmente abreviado OCR(por sus siglas en inglés), es la conversión mecánica o electrónica de imágenes escaneadas de texto impreso o mecanografiado en texto que pueda ser interpretado por una computadora. Es ampliamente utilizado como una forma de entrada de datos de algún tipo de fuente de datos original en papel, así sean documentos de pasaportes, facturas, extractos bancarios, recibos, tarjetas de visita, correo, o cualquier número de registros impresos\cite{arh-passport}\cite{arh-card}. Es un método común para la digitalización de textos impresos que pueden ser electrónicamente editados, buscados, almacenados de manera compacta, expuestos en linea y usados en procesos de máquinas como traducción de máquina, texto a voz, extracción de datos y minería de texto\cite{gbook}\cite{creaceed}. OCR es un campo de investigación en reconocimiento de patrones, inteligencia artificial y visión por computadora.
	
	\subsubsection{Aplicaciones}
		\begin{itemize}
			\item Reconocimiento automático de patentes\cite{arh-anpr}.
			\item Extraer información de una tarjeta de negocio a una lista de contacto \cite{x-root}.
			\item Hacer posible la búsqueda de imágenes electrónicas de documentos impresos. ej: \textit{Google Books} \cite{gbook}
			\item Convertir la escritura manual en tiempo real para controlar una computadora (\textit{pen computing})\cite{GKurt}\cite{sunnyside}.
			\item Tecnología de asistencia para usuarios ciegos y con deficiencias visuales \cite{creaceed}.
		\end{itemize}	
		
	\subsubsection{Componentes de un sistema OCR}
	
	A continuación se presentaran las diferentes etapas de un sistema OCR, como se puede apreciar en la figura \ref{fig: Sistema OCR}. Esta sección tiene como objetivo mostrar un panorama general de como funciona un sistema OCR, no tiene como objetivo adentrarse en detalles de implementación. En cada etapa se procederá a explicar brevemente cuales son las características importantes, haciendo incapié en la etapas de extracción de características y en la de clasificación y reconocimiento dado que son las etapas que se abordaron en este trabajo.
	
		\begin{figure}[htbp]
			\centering
			\fbox{ \includegraphics[scale=0.4]{img/OCR_pipeline_1.jpg} }
			\caption{Pipeline de un sistema OCR.}
			\label{fig: Sistema OCR}
		\end{figure}
	
		\paragraph{Pre-procesamiento} ~\\

		 El software de OCR aveces realiza un pre-procesamiento de las imágenes con el objetivo de tener más chances de un reconocimiento exitoso. Estas técnicas incluyen:
		  \begin{itemize}
		  	\item \textit{Enderezar} - Si el documento no está alineado debidamente cuando se escanea, puede necesitar posteriormente que se rote unos pocos grados en sentido horario o anti-horario con el objetivo de mantener el texto perfectamente vertical u horizontal \ref{fig: Enderezar}.
		  	\item \textit{Quitar manchas} - remover puntos negativos  y positivos, suavizar los bordes \ref{fig: Imagen con particulas} \ref{fig: Imagen suavizada}.
		  	\item \textit{Binarización} - Convertir la imagen de color o en escala de grises a blanco y negro (se llama "imagen binaria" justamente porque hay dos colores solamente). En algunos casos, esto es necesario para el algoritmo de reconocimiento de caracteres; en otros casos esta etapa se omite dado que el algoritmo tiene un mejor rendimiento con la imagen original \ref{fig: Binarizacion}.
		  	\item \textit{Eliminación de lineas} - Limpia las lineas y las cajas sin glifo \ref{fig: Eliminacion de lineas}.
		  	\item \textit{Análisis de disposición o "zoning"} - Identifica columnas, párrafos, títulos, etc. como bloques diferentes. Especialmente importante en tablas o disposiciones multi-columna \ref{fig: Zoning}.
		  	\item \textit{Detección de lineas y palabras} - Establece  la base para la forma del caracter y la palabra, separa las palabras si es necesario.
		  	\item \textit{Segmentación o aislación de caracter} - Múltiples caracteres que están conectados debido a artefactos de imagen deben ser separados; caracteres individuales que son separados en multiples piezas debido a artefactos deben ser conectados \ref{fig: Segmentacion}.
		  	\item \textit{Normalización de escala y relación de aspecto \ref{fig: Normalizacion y Suavizado}.}
		  \end{itemize}
		  
		\begin{figure}[htbp]
			\centering
			\subfloat[Imagen original torcida\label{fig: Imagen torcida}]{
				\fbox{ \includegraphics[scale=0.7]{img/skew_img.png} }
			}
			\subfloat[Imagen enderezada\label{fig: Imagen destorcida}]{
				\fbox{ \includegraphics[scale=0.7]{img/deskew_img.png} }
			}
			\caption{Imagen enderezada}
			\label{fig: Enderezar}
		\end{figure}	
			  
		\begin{figure}[htbp]
			\centering
			\fbox{ \includegraphics[scale=0.4]{img/rem_particles.png} }
			\caption[Imagen con partículas]{Comparación de caracteres con y sin partículas. 1. Imagen con partículas.}
			\label{fig: Imagen con particulas}
		\end{figure}
		
		\begin{figure}[htbp]
			\centering
			\fbox{ \includegraphics[scale=0.4]{img/smooth_char.png} }
			\caption{Imagen suavizada.}
			\label{fig: Imagen suavizada}
		\end{figure}
		
		\begin{figure}[htbp]
			\centering
			\subfloat[Imagen Original\label{fig: Imagen bin original}]{
				\fbox{ \includegraphics[scale=0.7]{img/bin_1.png} }
			}
			\subfloat[Imagen binarizada\label{fig: Imagen binarizada}]{
				\fbox{ \includegraphics[scale=0.7]{img/bin_2.png} }
			}
			\caption{Binarización de una imagen}
			\label{fig: Binarizacion}
		\end{figure}
		
		\begin{figure}[htbp]
			\centering
			\subfloat[Imagen Original\label{fig: Imagen lines original}]{
				\fbox{ \includegraphics[scale=0.7]{img/lines_1.png} }
			}
			\subfloat[Imagen con las lineas removidas\label{fig: Imagen sin lineas}]{
				\fbox{ \includegraphics[scale=0.7]{img/lines_2.png} }
			}
			\caption{Eliminación de lineas}
			\label{fig: Eliminacion de lineas}
		\end{figure}
		
		\begin{figure}[htbp]
			\centering
			\fbox{ \includegraphics[scale=0.3]{img/zoning.png} }
			\caption{Análisis de disposición o "zoning".}
			\label{fig: Zoning}
		\end{figure}

		\begin{figure}[htbp]
			\centering
			\fbox{ \includegraphics[scale=0.5]{img/segmentation.png} }
			\caption[Segmentación]{Segmentación. 1.Imagen Adquirida, 2.Región de interés, 3.Rectángulo del límite del carácter, 4.Carácter, 5.Artefacto, 6.Elemento, 7.Espaciado vertical, 8.Espaciado horizontal, 9.Espaciado del carácter.}
			\label{fig: Segmentacion}
		\end{figure}
		
		\begin{figure}[htbp]
			\centering
			\fbox{ \includegraphics[scale=0.4]{img/norm_and_smoothing.png} }
			\caption{Normalización y suavizado de un símbolo.}
			\label{fig: Normalizacion y Suavizado}
		\end{figure}
	
\newpage
		\paragraph{Ubicación y segmentación}	 ~\\
		
		La segmentación es el proceso que determina los componentes de una imagen. Es necesario ubicar las regiones del documento donde se han impreso los datos y distinguirlos de los gráficos y las figuras. Por ejemplo, cuando se realiza el ordenamiento de correo automático, la dirección debe estar localizada y separada de otras impresiones en el sobre como las estampillas o logos de empresa, anterior al reconocimiento.
		
		Aplicado al texto, la segmentación es la aislación de caracteres o palabras. La mayoría de los algoritmos de reconocimiento óptico de caracteres segmentan las palabras en caracteres aislados los cuales son reconocidos individualmente. Usualmente esta segmentación es realizada aislando cada componente conectado, es decir, cada área negra conectada. Esta técnica es fácil de aplicar, pero los problemas ocurren si los caracteres se tocan o si los caracteres están fragmentados y consisten de varias partes, si el texto tiene mucho ruido o se confunde con alguna imagen, entre otros. ~\ref{fig: Caracter quebrado}~\ref{fig: Caracteres unidos 1}~\ref{fig: Caracteres unidos 2}~\ref{fig: Caracteres en graffiti}~\ref{fig: Captcha}.

		\begin{figure}[htbp]
			\begin{minipage}{.3\linewidth}
				\centering
				\subfloat[Captcha\label{fig: Captcha}]{
					\fbox{ \includegraphics[scale=0.2]{img/degraded_1.jpg} }
				}
			\end{minipage}
			\begin{minipage}{.3\linewidth}
				\centering
				\subfloat[Carácter quebrado\label{fig: Caracter quebrado}]{
					\fbox{ \includegraphics[scale=0.3]{img/degraded_2.jpg} }
				}
			\end{minipage}
			\begin{minipage}{.3\linewidth}
				\centering
				\subfloat[Caracteres unidos por la tipografía\label{fig: Caracteres unidos 1}]{
					\fbox{ \includegraphics[scale=0.3]{img/degraded_3.jpg} }
				}
			\end{minipage}
			\begin{minipage}{.3\linewidth}
				\centering
				\subfloat[Carácter con linea superpuesta\label{fig: Caracter y linea}]{
					\fbox{ \includegraphics[scale=0.3]{img/degraded_4.jpg} }
				}
			\end{minipage}
			\begin{minipage}{.3\linewidth}
				\centering
				\subfloat[Caracteres unidos\label{fig: Caracteres unidos 2}]{
					\fbox{ \includegraphics[scale=0.3]{img/degraded_5.jpg} }
				}
			\end{minipage}
			\begin{minipage}{.3\linewidth}
				\centering
				\subfloat[Un graffiti como el de la imagen puede hacer pasar el texto como imagen.\label{fig: Caracteres en graffiti}]{
					\fbox{ \includegraphics[scale=0.2]{img/text_like_graphic.jpg} }
				}
			\end{minipage}
			\caption{Problemas de segmentación}
			\label{fig: Problemas de segmentacion}
		\end{figure}
		
	
		\paragraph{Extracción de características} ~\\
		
		El objetivo de la extracción de características es capturar los rasgos esenciales de los caracteres para formar vectores de características y es generalmente aceptado como uno de los problemas más difíciles en el reconocimiento de patrones. Se vuelve sencillo para el clasificador el clasificar entre clases diferentes teniendo en cuenta estas características \cite{PSH2011}. Acorde al trabajo de C. Y. Suen \cite{Suen86}, los rasgos de un carácter pueden ser clasificados en dos clases: características globales o estadísticas y características estructurales o topológicas.
		\begin{itemize}
			\item \textbf{Características globales o estadísticas}. \\
				Las características globales son obtenidas del arreglo de puntos que constituyen la matriz del carácter. Estas características pueden ser detectadas fácilmente en comparación con las características topológicas. Además no son afectadas demasiado por el ruido o la distorción en comparación a las topológicas. Un número de técnicas se utilizan para lograr la extración de características, estas son: \textit{momentos, zonificación, histogramas de proyección, n-tuplas y cruce y distancias}
				\begin{itemize}
					\item \textbf{Momentos}
					En este caso los diferentes puntos presentes en un carácter son utilizados como características. Heutte et al. \cite{Heutte98} decían que estos métodos son más comúnmente usados en el reconocimiento de caracteres.
					\item \textbf{Zonificación}
					De acuerdo con esta técnica, la matriz del carácter es dividida en pequeñas porciones o zonas. Las densidades de los píxeles en cada zona son calculadas y usadas como características. Este concepto fue sugerido por Hussain et al.\cite{Hussain72}.
					\item \textbf{Histogramas de proyección}
					Los histogramas de proyección nos da el número de píxeles negros en las direcciones horizontal y vertical de un área específica del carácter. El concepto fue introducido por M. H. Glauberman \cite{Glauberman56}. Los histogramas de proyección pueden ser vertical, horizontal, diagonal izquierda o derecha.
					\item \textbf{N-tuplas}
					De acuerdo con este método, la posición de los píxeles blancos o negros en la imagen de un carácter es considerado una característica. El método fue desarrollado por Tarling y Rohwer \cite{TR93}.
				\end{itemize}
			\item \textbf{Características estructurales o topológicas}. \\
				Estas características están relacionadas con la geometría del conjunto de caracteres. Algunas de estas características son concavidades y convexidades en los caracteres, el número de puntos finales, el número de agujeros en los caracteres, etc. Muchas investigaciones fueron realizadas con el objetivo de encontrar diferentes características estructurales.
				
				Rocha y Pavlidis \cite{RP94} propusieron un método para el reconocimiento de caracteres impresos del tipo multi-fuente. Para este método se han utilizado características estructurales tales como puntos singulares, arcos convexos y trazos.
				
				Otro método similar fue propuesto por A. Amin \cite{Amin2000}. Aquí se han usado características estructurales como el número de subpalabras, el número de picos en cada subpalabra, el número de curvas de cada pico, entre otros.
				
		\end{itemize}
		  
		  	En el presente trabajo, las características de una imagen son representadas por un vector binario que se obtiene de binarizar el descriptor HOG(concepto descripto en la sección \ref{sec: HOG}) de dicha imagen con un umbral pre-calculado. Este método de extracción de características se abordará más adelante en el capítulo 4 cuando se explique el pipeline de procesamiento.
	  
		\paragraph{Clasificación y Reconocimiento} ~\\
		
		La clasificación es el proceso de identificar cada carácter y asignarle la clase correcta. La clasificación se hace generalmente mediante la comparación de los vectores de características que corresponden al carácter de entrada con el representante de cada clase de carácter. Pero antes de hacer esto, el clasificador debe entrenarse con un conjunto de entrenamiento que represente a las diferentes clases de caracteres. Varios investigadores han propuesto métodos de clasificación dentro de los cuales se encuentran los métodos estadísticos, métodos sintácticos, comparación de plantillas, redes neuronales artificiales, métodos de núcleo~\cite{NNSJ}~\cite{SA96}~\cite{RYVY2010}.
		
		Como se procederá a explicar en detalle en el capítulo 3, el clasificador que se usa en este trabajo es Random Ferns. Las razones de su elección se basan en que es un clasificador multi-clase bastante eficiente y dado que el problema en cuestión es clasificar caracteres (representan 62 clases en total), su elección resulta apropiada.
		
		El aporte de este trabajo se basa en analizar la performance del clasificador Random Ferns cuando es entrenado con caracteres sintéticos. Si bien este enfoque fue abordado en \cite{wang}, en dicho trabajo no se puede apreciar como influye en la clasificación la cantidad de imágenes sintéticas usadas al entrenar el clasificador. En su trabajo, hacen uso de 1000 imágenes sintéticas por clase para entrenar el clasificador y realizan una comparación de performance entre el entrenamiento con imágenes sintéticas y el realizado con imágenes reales. Este trabajo va un paso más allá y busca ver si el número de imágenes sintéticas por clase influye realmente en la performance y además se busca analizar un nuevo enfoque en el cual se mezclan en diferentes proporciones imágenes reales con sintéticas para observar que tan efectivo resulta ser el clasificador en esas condiciones.
		
		\paragraph{Post-procesamiento} ~\\
		
		La precisión de OCR puede ser incrementada si la salida está restringida por un lexicón, que es una lista de palabras cuya aparición está permitida dentro del documento. Este puede ser, por ejemplo, todas las palabras del lenguaje español, o un lexicón mas técnico para un campo en particular. Esta técnica puede ser problematica si el documento contiene palabras que no están en el lexicón, como nombres propios.
		
		La salida puede ser texto plano o un archivo de caracteres, pero sistemas de OCR más sofisticados pueden conservar la disposición original de la pagina y producir, por ejemplo, un PDF con anotaciones que incluyan la imagen original de la página y una representación textual de búsqueda.
		
		Conocer la gramática del lenguaje que está siendo escaneado puede ayudar a determinar si la palabra es un verbo o un sustantivo, por ejemplo, permitiendo una gran precisión.
		
	\subsubsection{Errores típicos en OCR}
	
		La precisión de los sistemas de OCR es, en la práctica, directamente dependiente de la calidad de los documentos de entrada. Las principales dificultades se pueden clasificar como sigue:
		\begin{itemize}
			\item \textit{Variaciones en la forma}, debido a las variaciones en el estilo y los remates o serifas.
			\item \textit{Deformaciones}, causados por caracteres quebrados, manchados y moteados.
			\item \textit{Variación en el espaciado}, debido a los superíndices, subíndices, sesgo y la variable de espaciado.
			\item \textit{Mezcla de texto e imágenes}.
		\end{itemize}
		Estas imperfeccciones pueden afectar y ocasionar problemas las diferentes etapas del proceso de reconocimiento de un sistema OCR, resultando en errores de clasificación y rechazos.	
	
		
