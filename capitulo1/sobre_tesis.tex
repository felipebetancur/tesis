\subsection{Sobre el trabajo}

	Esta tesis presenta una reimplementación de una sección del trabajo presentado por Wang et. al. en \cite{wang}. En dicha sección, los autores buscan establecer un método para poder reconocer caracteres en imágenes naturales. Para esto proponen usar un clasificador llamado \textit{Random Ferns} (se explica en detalle en el próximo capítulo) para lograr este objetivo.
	
	Este trabajo tiene como finalidad analizar la performance en el reconocimiento de caracteres en imágenes naturales de dicho clasificador. Esto se realiza a través de diferentes experimentos que buscan evaluar diferentes conjuntos de imágenes. El primero es usando imágenes reales, de la misma manera que Wang et. al., con lo cual se busca comparar ambas implementaciones. Posteriormente, se busca analizar como influyen los caracteres sintéticos o fuentes en diferentes proporciones. Estas se alteran con el objetivo de intentar imitar a las imágenes reales y ver si se puede alcanzar o superar los resultados del primer conjunto. Por último y a diferencia de los autores originales, se propone entrenar al clasificador con  un conjunto nuevo que surge de mezclar en diferentes proporciones imágenes reales y sintéticas.