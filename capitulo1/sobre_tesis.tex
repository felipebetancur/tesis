\subsection{Sobre el Trabajo}

	Esta tesis presenta una reimplementación de una parte del trabajo presentado por Wang et. al. en \cite{wang}. En esta parte, los autores buscan establecer un método para poder reconocer caracteres en imágenes naturales. Para esto proponen usar un clasificador llamado \textit{Random Ferns}.
	
	Este trabajo tiene como finalidad principal analizar el impacto producido por el uso de datos sintéticos en la clasificación. Esto se realiza a través de ex\-pe\-ri\-men\-tos sobre diferentes conjuntos de imágenes. El primer ex\-pe\-ri\-men\-to, es usando imágenes reales, de la misma manera que Wang et. al., con lo cual se busca comparar ambas implementaciones. Posteriormente, se busca analizar cómo influyen el uso de los caracteres sintéticos en el entrenamiento. Estos se alteran con el objetivo de intentar imitar a las imágenes reales y ver si se puede alcanzar o superar los resultados del primer conjunto. Por último y a diferencia de los autores originales, se propone entrenar el clasificador con  un conjunto nuevo que surge de mezclar en diferentes proporciones imágenes reales y sintéticas.