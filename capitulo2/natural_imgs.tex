	Desde la aparición de las primeras fotografías, las personas han buscado ``inmortalizar'' escenas, objetos o personas, con el objetivo de, en el area de la ciencias, extraer información útil de las mismas que pueda ser utilizada para su analisis o estudio. Con el surgimiento de los primeros formatos digitales, la necesidad pasó por encontrar métodos automáticos que permitieran clasificar y reconocer elementos dentro de las imágenes. Desde reconocer texto manuscrito, texto en carteles publicitarios, patentes, personas, animales hasta identificar zonas con agua en una imagen satelital. La cantidad de potenciales aplicaciones que se pueden obtener es enorme. Es por eso que en campos de investigación como visión por computadora, este es un tema de interés. Sin embargo, la clasificación en imágenes naturales no es una tarea para nada sencilla. Por ejemplo, en el reconocimiento de texto, las imágenes naturales contienen muchas información ``extra'' que se tiene que tener en cuenta. Ya sea la existencia de otros objetos ajenos a la clasificación, es decir, elementos que no son texto como así también variaciones propias en las características de la misma imagen (se detallan más adelante en esta sección).
	
	Hoy en día se ha avanzado mucho en el área de la clasificación en escenas naturales. Se han desarrollado muchas aplicaciones como aquellas capaces de reconocer personas \cite{DT05}, hasta las que pueden identificar patentes \cite{DAB}. Si bien dichos avances muestran que es posible realizar lo mismo en diferentes ámbitos, en otros como el reconocimiento de texto sigue siendo un desafío (en contextos naturales).
	
	
\paragraph{Características} ~\\
	
	Las características en las imágenes naturales son muy variadas. Todas estas variaciones dificultan el trabajo sobre ellas por lo cual en general es necesario realizar un trabajo de pre-procesamiento antes de trabajar con las mismas. Las características que podemos encontrar en este tipo de imágenes son las variaciones de intensidad en la iluminación, la resolución, el ángulo en el que son tomadas las mismas, el fondo, las texturas, entre otros. Mas específicamente, dependiendo del objeto que se esté analizando, por ejemplo, texto, surgen más características como el tipo de fuente, el tamaño, la posición y orientación de los caracteres, la contaminación que pueda llegar a tener el texto por suciedad u oclusión, etc.
	
	Para poder analizar los caracteres en las imágenes naturales, uno de los enfoques que adoptan Wang et al. en \cite{wang} es el de trabajar con el descriptor HOG de cada imagen. Para poder entender que es un descriptor HOG (que se detalla más adelante), primero es necesario comprender el concepto de gradiente que se explica a continuación.
	
\paragraph{Gradientes} ~\\

	Sea $f(x_1,\dots,x_n)$ una función escalar de múltiples variables. El gradiente de $f$ representa la pendiente de la tangente del gráfico de $f$.  Mas precisamente, el gradiente apunta en la dirección donde se registra la mayor tasa de incremento de la función $f$ y su magnitud es la pendiente del gráfico de $f$ en esa dirección. Formalmente, es la generalización del concepto de derivada en funciones de múltiples variables.
		
	El gradiente de la función $f$ descripta anteriormente, es denotado como $\nabla f$ donde $\nabla$ (el símbolo nabla) denota el operador diferencial. El gradiente de $f$ es definido como el único campo vectorial cuyo producto punto con cualquier vector $v$ en cada punto $x$ es la derivada direccional de $f$ a lo largo de $v$. Es decir,
		 \begin{align*}
		 	(\nabla f(x))\cdot v = D_v f(x)
		 \end{align*}
		 
	En un sistema de coordenadas rectangular, el gradiente es el campo vectorial cuyos componentes son las derivadas parciales de $f$:
		 
		 \begin{align*}
		 	\nabla f(x) = \frac{\partial f}{\partial x_1}\mathbf{e}_1 + \cdots + \frac{\partial f}{\partial x_n }\mathbf{e}_n
		 \end{align*}
	donde los $\mathbf{e}_i$ son vectores unitarios ortogonales que apuntan en la dirección de coordenadas.

	En el procesamiento de imágenes, un gradiente es un cambio direccional en la intensidad o color de la imagen. El vector gradiente se forma combinando la derivada parcial de la imagen en las direcciones $x$ e $y$. Se puede expresar del a siguiente forma:
		\begin{align}
			\nabla I = \left( \frac{\partial I}{\partial x} , \frac{\partial I}{\partial y} \right)
		\end{align}	
		
	donde \textit{I} es la ``función intensidad''.Cuando determinamos la derivada parcial de $I$ respecto de $x$, determinamos la rapidez con que la imagen cambia de intensidad a medida que $x$ cambia. Para funciones continuas, $I(x,y)$, podemos expresarlo de la siguiente manera:
	\begin{align}
		\frac{\partial I(x,y)}{\partial x} = \lim_{\nabla x\rightarrow 0} \frac{I(x + \nabla x, y) - I(x,y)}{\nabla x}	
	\end{align}
	
	 El calculo de los gradientes de una imagen es útil ya que sirve, por ejemplo, para realizar detección de bordes de un objeto. La detección de bordes busca identificar puntos en una imagen en donde el brillo de la misma cambie de manera abrupta o, más formalmente, tenga discontinuidades. El propósito de esto es capturar eventos imporantes o cambios en las propiedades de una imagen. En este caso, después de que los gradientes han sido computados, los píxeles con alto valor de gradiente son elegido como posibles bordes. Los píxeles con el valor de gradiente más alto en la dirección del gradiente se convierten en píxeles de borde. Los gradientes, también pueden ser usados en aplicaciones que realizan reconocimiento de objetos o correspondencia de texturas \textbf{agregar referencias}.
	 
\paragraph{Descriptores de regiones} ~\\

	\RC{Completar}
