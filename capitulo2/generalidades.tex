\subsubsection{Características}

Las características en las imágenes naturales son muy variadas. Dentro de las mismas podemos encontrar las variaciones de intensidad en la iluminación, la resolución, el ángulo en el que son tomadas, el fondo, las texturas, entre otros. Mas específicamente, dependiendo del objeto que se esté analizando, por ejemplo, texto, surgen más características como el tipo de fuente, el tamaño, la posición y orientación de los caracteres, la contaminación que pueda llegar a tener el texto por suciedad u oclusión, etc. La infita variedad que es posible encontrar en este tipo de imágenes, dificultan el trabajo de reconocimiento sobre ellas por lo que, en general, es necesario realizar un pre-procesamiento antes de usarlas.

	En el caso del reconocimiento de texto, dada la gran cantidad de formas en que se puede encontrar la imagen de un carácter, es necesario encontrar algún método que extraiga las características mas representativas para poder distinguilo. Para poder analizar los caracteres en las imágenes naturales, uno de los enfoques que adoptan Wang et al. en \cite{wang} es el de trabajar con el descriptor HOG de cada imagen. Para poder entender que es un descriptor HOG (que se detalla en \ref{subsection:hog}), primero es necesario comprender el concepto de gradiente que se explica a continuación.
