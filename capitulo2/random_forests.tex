\subsubsection{Random Forest}

	\paragraph{Árbol aleatorio} ~\\
	
	Un árbol aleatorio es un árbol construido a través de un proceso no determinístico. A diferencia de los algoritmos \textit{ID3} y \textit{C4.5}, explicados en la sección anterior, donde la elección del valor del nodo se da a través de la función de entropía, en los árboles aleatorios cada nodo del árbol se construye a partir de un proceso aleatorio que le asigna su valor. Dentro de los árboles aleatorios más comunes, podemos encontrarnos a  los \textit{árboles binarios} los cuales son construido insertando un nodo a la vez de acuerdo a una permutación aletoria.

	\paragraph{Random Forest} ~\\

	\textit{Random Forest} o \textit{bosque aleatorio} es un método de aprendizaje conjunto o\textit{ ensemble learning} para la clasificación o regresión. En el caso de clasificación, consiste en una colección de clasificadores con estructura de árbol $\{h(x,\Theta_k), k = 1,\dots\}$, donde $\{\Theta_k\}$ son vectores aleatorios independientes e identicamente distribuidos ($\Theta_k$ representa los parámetros para la construcción del $k$-esimo árbol) y $h(x,\Theta_k)$ es un clasificador donde $x$ es un vector de entrada. Luego, dada una entrada $x$, cada árbol emite un único voto para la elección de la clase más popular para $x$ \cite{Breiman01}.

	El algoritmo de entrenamiento para random forest aplica la técnica general de \textit{bootstrap aggregating}(agregación bootstrap) o \textit{bagging}(embolsado). Esta técnica fue desarrollada por Breiman en 1996 ~\cite{LBreiman96} y es un método para generar múltiples versiones de un predictor y usar estos para obtener un predictor agregado. Para tener una idea más clara del concepto, consideremos el siguiente ejemplo. Supongamos que tenemos $L = \{ (x_n,y_n), n = 1,\dots, N \}$ el cual es un conjunto de aprendizaje donde $x_n$ son valores de entrada e $y_n$ son etiquetas de clase (o valores numéricos en el caso de regresión). Supongamos también que tenemos una forma de generar un predictor de la forma $\varphi(x,L)$ a partir de $L$ tal que $ \varphi(x,L) = y $. Por último, supongamos que nos dan una secuencia de conjuntos de aprendizaje $\{ L_k \}$ donde cada uno consiste en $N$ observaciones independiente bajo la misma distribución de $L$. El objetivo es que usando $\{ L_k \}$ se obtenga un predictor mejor que el establecido anteriormente $\varphi(x,L)$ que usa solamente un conjunto de entrenamiento. La única restricción, es que se nos obliga a trabajar solamente con el conjunto de predictores $\{ \varphi(x, L_k)\} $.

	Para resolver este problema, Breiman estableció el siguiente criterio. Si la respuesta $y$ era un valor numérico luego, se reemplaza a $\varphi(x,L)$ por el promedio del conjunto de predicotres $ \{ \varphi(x, L_k) \} $ sobre $k$. Es decir, $\varphi_A(x) = \mathbb{E}_L\varphi(x,L)$ donde $\mathbb{E}_L$ denota la esperanza sobre $L$, y el subíndice $A$ en $\varphi_A$ denota la agregación. En cambio, si $\varphi(x,L)$ predecía una etiqueta de clase $j \in \{ 1,\dots, J \} $, luego un método para agregar todos los predictores era a través del voto. Es decir, $N_j = \{ k;\varphi(x, L_k) = j \}$ y se toma a $\varphi_A(x) = argmax N_j$.

	El problema principal, es que generalmente se tiene sólo un conjunto de aprendizaje $L$. Para esto, Breiman considera que se puede imitar el procedimiento anterior tomando repetidas muestras bootstrap $\{ L^{B} \}$ a partir de $L$, y formar $\{ \varphi(x, L^{B}) \}$. Formalmente dado $L$, una muestra boostrap $\{ L^{B} \}$ se genera tomando $b$ muestras de $L$ uniformemente con reemplazo. Dado que hay reemplazo, se puede dar el caso de que se repitan elementos en $\{ L^{B} \}$. Si $y$ es numérica luego, toma $\varphi_B$ como
	\begin{align*}
		\varphi_B(x) = av_B\varphi(x,L^{B}).
	\end{align*}

	donde \textit{av} denota la media sobre el conjunto de predictores. Si $y$ es una etiqueta de clase, luego el conjunto  $\{ \varphi(x, L^{B}) \}$ vota para formar $\varphi_B(x)$. El autor a este procedimiento lo llama  \textit{bootstrap aggregating} o \textit{bagging}.

	Cabe aclarar, que cada $L_i \in \{ L^{B} \}$ consta de $N$ muestras obtenidas al azar, pero con reemplazo, de $L$. Cada $(x_n, y_n)$ puede aparecer repetido una cierta cantidad de veces o no en $L_i$.

	Se puede aplicar \textit{bagging} para generar un algoritmo para árboles de decisión o regresión. Dado un conjunto de aprendizaje $L$ como el explicado anteriormente, la técnica de bagging selecciona repetidamente muestras de bootstrap del conjunto de aprendizaje $L$ y ajusta los árboles a estas muestras:

	Para $b=1$ hasta $B$:
	\begin{itemize}
		\item Se realiza un muestreo, con reemplazo, de $n$ ejemplos de entrenamiento a partir de $L$; llamemos a esta muestra $L_b$ (muesta bootstrap).
		\item Entrena un árbol de decisión o regresión $f_b$ a partir de $L_b$.
	\end{itemize}

	Después del entrenamiento, las predicciones para ejemplos no vistos $x'$ se pueden realizar a través del voto de todos los predicciones de la siguiente manera:
	$$N_j = \{ k;\varphi(x', L_k) = j \}$$
	y se toma a
	$$\varphi_A(x') = argmax N_j$$
	o tomar el promedio en caso de árboles de decisión: $\bar{f} = \frac{1}{B}\sum_{b=1}^B\bar{f_b}(x')$.

	En el algoritmo de arriba, B es un parámetro libre que indica la cantidad de árboles predictores que se van a emplear. Típicamente, algunos cientos o miles de árboles son usados, dependiendo del tamaño y naturaleza del conjunto de entrenamiento.

	El procedimiento anterior describe el algoritmo original de \textit{bagging} para árboles. Desafortunadamente, volver a correr el mismo algoritmo de aprendizaje en diferentes subconjuntos de los datos puede resultar en predictores altamente correlacionados, lo cual limita la reducción de varianza. La técnica conocida como \textit{random forest}, construye árboles basados en un subconjunto de variables de entrada elegidas al azar.

	Cada árbol es construido siguiendo el siguiente algoritmo:
	\begin{itemize}
		\item Si el número de muestras en el conjunto de entrenamiento es $P$, muestrear $N$ casos aleatoriamente - pero con reemplazo, a partir de los datos originales. Esta muestra va a ser el conjunto de entrenamiento para la construcción del árbol.
		\item Si hay $M$ variables de entrada, se especifica un número $m<<M$ (constante durante el crecimiento del bosque o forest) tal que en cada nodo se seleccionen $m$ variables al azar de las $M$. Posteriormente se eligen entre las $m$ variables aquellas que mejor dividan al nodo, es decir, aquellas que generen al final un árbol compacto y simple.
		\item Cada árbol se construye hasta su máxima extensión posible. No hay pruning(poda).
	\end{itemize}
	Las ventajas random forests son:
	\begin{itemize}
		\item Correr eficientemente en grandes bases de datos.
		\item Poder manejar cientos de variables entrantes sin excluir ninguna.
		\item Dar estimaciones de qué variables son importantes en la clasificación.
		\item Ofrecer un método experimental para detectar las interacciones de las variables.
	\end{itemize}
	Las desventajas de este algoritmo se pueden resumir en estos puntos:
	\begin{itemize}
		\item A diferencia de los árboles de decisión, la clasificación hecha por random forests es difícil de interpretar por el hombre.
		\item Si los datos contienen grupos de atributos correlacionados de relevancia similar para el rendimiento, entonces los grupos más pequeños están favorecidos por sobre los grupos más grandes.
	\end{itemize}
	
