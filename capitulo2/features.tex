\subsubsection{Vectores de características} \label{subsection:feature}
	
	Hasta el momento, se ha introducido la idea general de qué es un clasificador probabilístico. Se detalló que estos clasificadores toman una ``entrada \textit{X}'' y le asignan una probabildiad de que dicha entrada pertenezca a cada una de las clases asociadas. El proceso interno para realizar esto varia entre clasificadores, sin embargo, un concepto necesario para comprender cómo trabajan cuando ``procesan'' una entrada es el de \textit{característica}.

	Un \textit{característica} o \textit{feature}, es un aspecto o cualidad distintiva de un objeto (clase de interés). Las características son importantes dado que al representar los aspectos o cualidades más significativas de un objeto, facilitan el reconocimiento posterior de objetos similares \cite{OIVIND95}. Esto es fundamental por ejemplo, en los esquemas de clasificación en el aprendizaje supervisado ya que permite, dada una muestra desconocida, identificar a qué clase pertenece si anteriormente sabemos las características particulares de cada clase (a partir de instancias o muestras analizadas con anterioridad). Esto conlleva a que si se realiza una buena selección de características, impacte positivamente en la precisión de los clasificadores y por ende en el reconocimiento. Por ejemplo, en los algoritmos de detección de spam, las características pueden incluir el lenguaje en el que está escrito el email, la ausencia o presencia de ciertos encabezados, la corrección gramatical del texto, entre otros~\cite{SpamPaper}.

	Un \textit{vector de características} o ``feature vector'' se lo puede definir como un conjunto de características que definen a un objeto. Dicho objeto (en adelante $X$) es representado por un vector $D$-dimensional tal que  $X=(x_1,\dots,x_D)$ donde los $x_i$ representan a las características del objeto. En general, $x_i, i=1,\dots,D$ son numéricas dado que dicha representación facilita el análisis estadístico y el procesamiento de datos.

	En base a lo explicado anteriormente, se presenta a continuación al primer clasificador probabilístico que va a servir de ayuda para comprender los siguientes clasificadores.
		
	
