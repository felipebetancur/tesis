\subsubsection{Vectores de características} \label{subsec:feature}

	Un característica o feature, es un aspecto o cualidad distintiva de un objeto (clase de interés). Las características son importantes dado que al representar los aspectos o cualidades mas significativas de un objeto, facilitan el reconocimiento posterior de objetos similares. Esto es fundamental en el aprendizaje supervisado ya que permite dada una muestra desconocida, identificar a que clase pertenece si anteriormente sabemos las características particulares de cada clase (a partir de instancias o muestras analizadas con anterioridad). Esto conlleva a que si se realiza una buena selección de características, impacte positivamente en la precisión de los clasificadores y por ende en el reconocimiento. Por ejemplo, en los algoritmos de detección de spam, las características pueden incluir el lenguaje en el que esta escrito el email, la ausencia o presencia de ciertos encabezados, la corrección gramatical del texto, entre otros~\cite{SpamPaper}.

	Un vector de características o feature vector, es un  vector $n$-dimensional de características numéricas que se diseña de forma tal de conjeturar propiedades características de los objetos. Muchos algoritmos en machine learning requieren la representación numérica de los objetos, dado que dichas representaciones facilitan el análisis estadístico y el procesamiento.
		
	
		
	
