\paragraph{Descriptor SIFT} ~\\
	\textbf{Reformular}
	Scale-invariant feature transform o SIFT(por su sigla en inglés) es un algoritmo en visión por computadora desarrollado por David G. Lowe ~\ref{LoweDavid99} para detectar y describir las características locales de una imagen. Estas características, como se explico en secciones anteriores, sirven para identificar a una clase de interés u objeto cuando se la trata de localizar dentro de una imagen donde hay varios objetos. Los descriptores SIFT tienen la característica de no alterarse ante un cambio en la escala o en la orientación y son robustos en el sentido de que pueden detectar objetos si estos están desordenados o si los mismos están parcialmente ocultos. Además, son parcialmente invariantes a las transformaciones afines y a los cambios en la iluminación.
	
	En su trabajo, David G. Lowe explica que una forma de extraer estas características es tomando un enfoque donde se realiza un filtro en cascada en el cual las operaciónes mas costosas son aplicadas solamente en ubicaciones que pasan la prueba inicial. Dicho enfoque busca minimizar el coste de extracción de estas características. A continuación, se presenta una breve descripción, extraida del trabajo de D. G. Lowe, de las etapas de computación usadas para generar el conjunto de características de la imagen:
	\begin{enumerate}
		\item \textbf{Detección extrema de escala-espacio:} La primera etapa de computación, busca en todas las escalas y ubicaciones de la imagen. Básicamente, lo que se busca es encontrar ubicaciones y escalas en la imagen que puedan volver a ser identificadas en la misma aún si esta cambia de escala o punto de vista. Esto se puede lograr buscando característica estables a través de todas las escalas posibles de la imagen.
		\item \textbf{Ubicación del punto clave:} En esta etapa, se busca ajustar un modelo en la ubicación de cada candidato obtenidos durante la primera etapa con el objetivo de determinar la ubicación y la escala. Los puntos clave son seleccionados basados en las medidas de su estabilidad.
		\item \textbf{Asignación de orientación:} Se asignan una o más orientaciones en la ubicación de cada punto clave basado en las direcciones del gradiente de la imagen local. Todas las operaciones futuras se realizan en el dato de la imagen que ha sido transformada en relación a la orientación asignada, la escala y la ubicación de cada característica, proporcionando de este modo invarianza a estas transformaciones.
		\item \textbf{Descriptor del punto clave:} Se miden los gradientes de la imagen local en la escala seleccionada en la región alrededor de cada punto clave. Estos son transformados en una representación que permite la distorción local de la forma y los cambios en la iluminación.
	\end{enumerate}
	
	Un aspecto importante de este enfoque es que se genera un gran número de características que cubren densamente la imagen sobre toda la gama de escalas y ubicaciones.