\paragraph{Descriptor SIFT} ~\\

	\textit{Scale-invariant feature transform} o SIFT (por su sigla en inglés) es un algoritmo en visión por computadora desarrollado por David G. Lowe \cite{LoweDavid04} para detectar y describir las características locales de una imagen. Estas características, como se explico en secciones anteriores, sirven para identificar a una clase de interés u objeto cuando se la trata de localizar dentro de una imagen donde hay varios objetos. Los descriptores SIFT tienen la característica de no alterarse ante un cambio en la escala o en la orientación. Además, son parcialmente invariantes a las transformaciones afines y a los cambios en la iluminación.
		
	Para obtener estas características, se realiza un muestreo de las orientaciones y magnitudes del gradiente de la imagen. En el trabajo de Lowe, este muestreo se realiza sobre regiones de $16 \times 16$ alrededor del punto de interés. Se analizan las muestras de cada región de $16 \times 16$ formando histogramas de orientaciones resumiendo el contenido en sub-regiones de $4 \times 4$. Cada uno de los histogramas se compone de 8 orientaciones. Por lo tanto se obtienen 16 histogramas respecto de las orientaciones de los puntos de cada región para cada uno de los puntos de interés. Finalmente el descriptor de cada punto de interés está formado por un vector que contiene los valores de las 8 orientaciones de los $4 \times 4$ histogramas, con lo cual se obtendría un vector de 128 elementos.

	Dado que los cambios en la iluminación afectan en mayor medida a la magnitud del gradiente y no a la orientación, se busca una representación de esta magnitud que minimice estos efectos. El objetivo es modificar al vector para darle cierta robustez frente a cambios en la iluminación. Para eso se lleva a cabo un proceso de normalización, donde los cambios en la luminosidad no afecta a los valores del gradiente. Finalmente, se limita el valor de cada componente de magnitud de gradiente a un valir máximo para que tenga un mayor peso la orientación frente a la magnitud del gradiente. Siguiendo los parámetros de Lowe en \cite{LoweDavid04}, el valor del umbral es de 0,2. Luego se vuelve a normalizar a una amplitud de unidad.

	
	