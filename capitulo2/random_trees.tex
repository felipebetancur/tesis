\paragraph{Árbol aleatorio}

	Para comprender que es un árbol aleatorio, es necesario comprender que es un \textit{proceso estocástico}.

	Un proceso estocástico, es un proceso que se caracteriza por su indeterminación. Dicho de otra manera, la evolución de este, puede ir por muchos caminos posibles, incluso si conocemos el punto de partida (o condición inicial). Se diferencian de los procesos determinísticos, ya que estos últimos evolucionan de una sola manera, es decir, que no involucran la aleatoriedad en el desarrollo de los futuros estados del mismo.

	Teniendo en cuenta este concepto luego, podemos decir, que un árbol aleatorio es un árbol construido a través de un proceso estocástico. Es decir, cada nodo del árbol se construye a partir de un proceso aleatorio que le asigna su valor. Dentro de los árboles aleatorios más comunes, podemos encontrarnos a  los \textit{árboles binarios} los cuales son construido insertando un nodo a la vez de acuerdo a una permutación aletoria. En la siguiente sección, se procederá a explicar el árbol aleatorio \textit{random forest} el cual es un clasificador que se caracteríza por elegir subconjuntos de variables al azar para la construcción de cada árbol.

\JS{no dice mucho}