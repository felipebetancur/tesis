\subsection{Métricas y Protocolo de Evaluación}

\FL{Explicar cómo se divide el dataset en train y test y qué métricas se calculan sobre el test (accuracy por ejemplo).}
	El dataset utilizado para los experimentos es \textit{Chars74k-15} que es una variación del dataset \textit{Chars74k} para poder reproducir los experimentos de Wang et al. en \cite{wang}. El mismo, consta de un conjunto de entrenamiento y uno de prueba con 15 imágenes por clase cada uno (de 62 clases en total). En \textit{Chars74k} las imágenes reales de los caracteres están divididas en 2 grupos: las imágenes buenas y las malas. La diferencia radica, básicamente, en la calidad de la imagen provista. Las imágenes son de diferentes tamaño pero en el presente trabajo se las procesa para que tengan las mismas dimensiones. Tanto el conjunto de prueba como el de entranamiento tienen una mezcla de imágenes buenas y mala y ambos conjuntos son disjuntos.  Lo que se busca calcular sobre el conjunto de evaluación, es el grado de precisión del clasificador al haber entrenado al mismo con el conjunto de entrenamiento descripto.