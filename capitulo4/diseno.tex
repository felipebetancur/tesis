\subsection{Diseño de los experimentos}

	Dada la gran cantidad de parámetros que hay en el sistema, la cantidad de experimentos para encontrar la mejor configuración es extensa.

        \paragraph{Generación de datos sintéticos}

	Los parámetros utilizados durante la creación de las imágenes sintéticas son, como se explicó en el capítulo anterior, escala, rotación, blur, ruido Gaussiano e inclinación. Dado que hay un abanico bastante grande de valores para asignarles a estas transformaciones, se decidió por asignarle a cada una un rango de valores razonable. Dado que los autores en \cite{wang} no especifican el rango de valores para las transformaciones que utilizan, se procedió a establecer los siguientes intervalos para todas las transformaciones utilizadas:
	
	\begin{itemize}
		\item Transvección: $n \in [-0.20 ; 0.20]$
		\item Suavizado gaussiano (blur): $\sigma \in [0 ; 2]$
		\item Escala: $a=d \in [0.8; 1.25]$
		\item Rotación (en radianes): $\theta \in (-0.1; 0.1)$
		\item Ruido Gaussiano: $\sigma \in [1; 30]$
	\end{itemize}
	
	Por el hecho de no contar con una replica exacta del conjunto de datos sintéticos usados por Wang et al., los conjuntos generados con estos valores van a ser distintos a los originales y por ende la comparación de resultados  en \ref{subsection:resultados} va a estar influida por la forma en que se generaron tales conjuntos.
	
	\paragraph{Extracción de características con HOG}

	Posteriormente tenemos los parámetros propios que utiliza HOG para extraer las características de cada imagen. HOG hace uso de dos parámetros, la cantidad de \textit{orientaciones} y la cantidad de \textit{celdas por bloque}. Como se explicó en el capítulo anterior (\ref{subsection:hog}), una imagen se divide en regiones llamadas celdas. Dentro de cada celda, se realiza el cálculo de las orientaciones y se genera un histograma. Posteriormente, se agrupan las celdas en un bloque y se normalizan los histogramas. En los experimentos realizados, se consideran los siguiente conjuntos de valores para los parámetros:
	
	\begin{itemize}
		\item Orientaciones: $ o \in \{8; 9\}$
		\item Celdas por bloque: $ cpb \in \{4; 9\}$
	\end{itemize}
	
	Por ejemplo, si a una imagen de 48x48 píxeles en blanco y negro se le extrae el descriptor HOG utilizando como parámetros 8 orientaciones y 9 celdas por bloque se obtendría un descriptor de tamaño $1152$. Si se utilizaran 4 celdas por bloque la dimensión del descriptor resultante sería de $800$.
	
	 La elección de estos parámetros se debe a que reducen la tasa de errores de clasificación. Un análisis similar se puede encontrar en \cite{DT05} donde los autores (quienes crearon HOG) analizan la mejor configuración para resolver el problema de detección de personas. Si bien el problema que se busca resolver en este trabajo es totalmente diferente, los parámetros que ellos usan para HOG muestran buenos resultados en este problema también.

	\paragraph{Binarización}
	
	En la binarización, se puede establecer la dimensión del vector final. Este es otro parámetro (para los experimentos con imágenes sintéticas y mixtas) con el que se trabaja y que influye en los resultados. Los valores con los que se trabaja son los siguientes:

	\begin{itemize}
		\item Dimensión del vector final: $ d \in \{ 240; 480; 1080; 2040;  4080 \}$
	\end{itemize}
	
	La elección de estas dimensiones está directamente relacionada con el siguiente parámetro que es la cantidad de ``\textit{grupos por vector}'', por lo cual estas dimensiones tienen que ser divisibles por el número de los grupos.
	
	El último parámetro a destacar en la binarización es el método aplicado al momento de generar el umbral de binarización. Como se podrá ver en la subsección de resultados, es muy grande el impacto de este parámetro en la precisión final del clasificador. En \cite{wang}, los autores no especifican que método usan para poder binarizar los vectores, por ende, propusimos 4 métodos como alternativa:

	\begin{itemize}
		\item Media
		\item Mediana
		\item Bootstrap
		\item Exponencial
	\end{itemize}
	
	
	\paragraph{Entrenamiento}

	En la etapa de entrenamiento hay 2 parámetros: la cantidad de \textit{bits por grupo}, que determina finalmente la cantidad de grupos en la que se divide cada vector, y \textit{$\alpha$} que es un parámetro de inicialización para las tablas. La cantidad de grupos en las que se divide un vector impacta en el tamaño de las tablas y en la clasificación posterior. Por ejemplo, en el peor de los casos, si consideramos vectores binarios de tamaño $4080$ y grupos de $4080$ bits (1 solo grupo), luego se necesitaría una tabla con $2^{4080}$ entradas la cual es imposible de almacenar en memoria y mucho menos analizar. La cantidad de bits denota la cantidad de dimensiones de cada grupo, por lo cual  la cantidad de grupos está dado por la división entre la dimensión total del vector y la dimensión del grupo.

	El parámetro \textit{$\alpha$} es necesario ya que las tablas no se pueden inicializar con el valor $0$. Esto es debido a que, como se explicó en la sección \ref{subsection:ferns}, al momento de evaluar una imagen de prueba, puede darse el caso de que se acceda a una posición de la tabla que nunca fue alcanzada durante el entrenamiento por lo cual la probabilidad resultante se hace 0 y es un inconveniente para los cálculos posteriores. Este parámetro se obtiene de la ecuación \ref{eq:Laplace-Smoothing} (que es usada durante el entrenamiento del clasificador) y está representada por el valor $N_r$ en la misma. Valores muy grandes de $\alpha$ ($>>1$) aplicados a la ecuación \ref{eq:Laplace-Smoothing} hacen que la misma tienda a una distribución uniforme si el conjunto de entrenamiento no es muy grande lo cual afecta a la clasificación.

	\begin{itemize}
		\item bits por grupo: $ bpg \in \{ 1; 2; 4; 8; 10; 12 \}$
		\item $\alpha$ $ \alpha \in \{ 0.01; 0.1; 1 \}$
	\end{itemize}
