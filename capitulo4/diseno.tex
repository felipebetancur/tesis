\subsection{Diseño \JS{de experimentos}}
	\begin{itemize}
		\item Aca debería ir una descripción de los parámetros a usar en los experimentos con una explicación de lo que se espera obtener.
		\begin{itemize}
			\item Explicar para qué sirve cada parámetro
			\item Explicar los valores que se van a usar para cada parámetro
		\end{itemize}
		\item Explicar que datasets se van a crear para los experiementos
		\begin{itemize}
			\item Imagenes Reales (CHAR74K, 15img por clase)
			\item Imágenes sintéticas (50/100/500/1000/2000/3000/7000/10000 img por clase)
			\item Imágenes reales + sintéticas ((15+15)/(15+30)/(15+45)/(15+60)/(15+100)... imgs por clase). Cuando pongo ($x+y$), $x$ representa la cantidad de img reales e $y$ la cantidad de sintéticas. Quiero ver hasta que punto aumenta la performance de clasificación mezclando en diferentes proporciones sintétitcas + reales.
		\end{itemize}
	\end{itemize}
