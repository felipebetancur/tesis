\subsection{Conjuntos de Entrenamiento}

	En esta sección se va a discutir los diferentes conjuntos de entrenamientos usados en los experimentos. En las evaluaciones se trabaja básicamente con 3 conjuntos.
	
	El primero, es el compuesto sólo por imágenes de caracteres segmentados de escenas naturales.
	
	El segundo conjunto es el armado con imágenes sintéticas, cuyo proceso de creación fue explicado en detalle en la sección \ref{subsection:impl_propia}. El grupo de imágenes de fuentes que se usa para crear los caracteres sintéticos fue extraído del dataset mencionado anteriormente en \ref{subsection:evaluacion} y consiste en 1000 imágenes por clase. De este conjunto se crearon 6 variaciones diferentes, con el objetivo de observar el impacto que tiene la cantidad de caracteres sintéticos al momento de clasificar imágenes de caracteres naturales. La cantidad de muestras por clase (variaciones) en cada conjunto de entrenamiento es el siguiente:
	\begin{itemize}
		\item Muestras por clase : $ mpc \in \{ 50,100,250,500,1000,2000\}$
	\end{itemize}

	El tercer tipo está compuesto por una combinación de los dos primeros conjuntos. Se creó con la finalidad de observar el impacto en la clasificación de integrar en diferentes proporciones imágenes sintéticas y reales al conjunto de entrenamiento. Dada la escasa cantidad de imágenes naturales para el entrenamiento (15 en total por cada clase) se busca complementar esto integrando datos sintéticos. Por consiguiente se decide crear 9 conjuntos de entrenamiento donde en cada uno se incorporan diferentes proporciones de imágenes sintéticas a las 15 reales que existentes. La proporción de cada conjunto de entrenamiento es de la siguiente manera: 
	
	$$(15, x) x \in \{0,3,8,15,30,75,150,300,750,1500 \}$$
	
	El primer elemento de la tupla $(15, x)$ corresponde a la cantidad de imágenes reales, cuyo valor es fijo. El segundo elemento de la tupla hace referencia a la proporción de caracteres sintéticos. Es fácil observar que a medida que aumentamos la proporción, la cantidad de caracteres naturales se va volviendo cada vez más despreciable. El hecho de no considerar imágenes sintéticas inicialmente $(15,0)$ tiene como finalidad el observar el cambio que se produce desde el inicio cuando no hay caracteres sintéticos en el entrenamiento. Los resultados se analizan en la última sección del presente capítulo.
	
