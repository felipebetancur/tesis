\subsection{Arquitectura del sistema}
\label{subsection:impl_propia}

	En las próximas subsecciones, se procederá a explicar cuestiones de implementación que se tuvieron en cuenta, en este trabajo, al momento de resolver el problema del reconocimiento de caracteres.

	\subsubsection{Reconocimiento de caracteres}
		\label{subsubsection:recon-caracteres}
	
	
	\paragraph{Pipeline de procesamiento} ~\\

			La implementación que se realizó en este trabajo, está basada en un pipeline similar al que realizaron Wang et al. que comprende de dos instancias: la instancia de entrenamiento y la instancia de evaluación. La primera esta constituida por la generación de datos sintéticos que forman el conjunto de entrenamiento, la extracción de las características, la binarización y el entrenamiento del clasificador. La segunda instancia consiste en la evaluación del clasificador que abarca desde la extracción y binarización de las características del conjunto de prueba hasta la evaluación del clasificador para cada imagen de este conjunto. Los pipelines se pueden apreciar en las figuras \ref{fig: Pipeline entrenamiento} y \ref{fig: Pipeline evaluacion}.

			\begin{figure}[htbp]
				\centering
				\centerline{
					\includegraphics[scale=0.4]{img/pipeline_entrenamiento.jpg}
				}
				\caption[Pipeline de entrenamiento]{La figura representa el pipeline de entrenamiento que se sigue en el trabajo a la hora de entrenar el clasificador.}
				\label{fig: Pipeline entrenamiento}
			\end{figure}
			
			\begin{figure}[htbp]
				\centering
				\centerline{
					\includegraphics[scale=0.4]{img/pipeline_evaluacion.jpg}
				}
				\caption[Pipeline de evaluación]{La figura representa el pipeline de evaluación que se sigue en el trabajo al momento de evaluar el clasificador.}
				\label{fig: Pipeline evaluacion}
			\end{figure}
			
		\paragraph{Conjunto de entrenamiento} ~\\
		
			La primera etapa de este pipeline de procesamiento consiste en la obtención de un conjunto de entrenamiento para utilizar en el entrenamiento del clasificador. Se destacan dos grupos de imágenes al momento de entrenar, el de caracteres segmentados de escenas naturales y el de caracteres sintéticos. A continuación se procederá a explicar como se genera el grupo de datos sintéticos.
			

		\paragraph{Generación de datos sintéticos} ~\\

			Supongamos que se tiene un conjunto limitado de imágenes reales de caracteres. Uno querría poder trabajar con más imágenes, sin embargo, el tiempo y el esfuerzo que conlleva la recolección de este tipo de datos lo hacen una tarea poco práctica. Se busca una forma de automatizar la forma en que se pueden obtener este tipo de datos. Es por eso que el objetivo detrás de la generación de los datos sintéticos es agregar vistas de los caracteres que no están reflejadas en el conjunto de imágenes reales original pero que son plausibles de ser observadas en la realidad.
			
			Inicialmente, se tiene como base un conjunto que contiene imágenes de caracteres de diversas tipografías generadas de forma artificial. Lo que se busca, es aplicar a cada imagen un conjunto de transformaciones afines aleatorias con el objetivo final de obtener imagenes nuevas cuyas apariencias sean lo más cercanas a una real. Una transformación afín consiste en una transformación lineal seguida de una traslación. 
			
			Existen 2 tipos de transformaciones: \textit{geométricas} y \textit{fotométricas}. Las primeras son de la forma $x\mapsto Mx + b$, donde $M$ es una transformación lineal y $b$ es un vector. Se usan las multiplicaciones entre matrices para representar las transformaciones lineales y la suma de vectores para representar las traslaciones. Mediante matrices ampliadas, sin embargo, es posible representar ambos tipos de transformaciones.	La matriz $M$ presentada anteriormente es una matriz $2\times 2$ y permite realizar transformaciones en imágenes de 2 dimensiones. En este trabajo, para generar los datos sintéticos se hacen uso de las siguientes transformaciones geométricas:
			
			\paragraph{Rotación}
			
				La rotación es una transformación que consta en rotar la imagen en un ángulo $\theta$ en torno al origen, el centro de la imagen. El parámetro usado para el mismo son los radianes y las diferentes variaciones se pueden apreciar en la Figura \ref{fig: Transformacion Afin - Rotacion}. Teniendo en cuenta la definición formal de una transformación afín anterior, la rotación se puede representar formalmente de la siguiente manera:

			\begin{equation*}
					M =  
					\begin{bmatrix}
						cos(\theta) & -sin(\theta) \\
						sin(\theta) & cos(\theta)  \\
					\end{bmatrix}
					b =
					\begin{bmatrix}
						0 \\
						0 \\
					\end{bmatrix}	
			\end{equation*}

	donde se puede observar el parámetro $\theta$ que representa el grado de inclinación y el vector $b$ es nulo por lo cual no hay traslación alguna.

		\begin{figure}[htbp]
			\centering
			\subfloat[\label{fig: sintetica original}]{
				\fbox{ \includegraphics[scale=1]{img/transformaciones/original.png} }
			}
			\subfloat[\label{fig: Imagen rad 0.5}]{
				\fbox{ \includegraphics[scale=1]{img/transformaciones/rotation0,5.png} }
			}
			\subfloat[\label{fig: Imagen rad 1}]{
				\fbox{ \includegraphics[scale=1]{img/transformaciones/rotation1.png} }
			}
			\subfloat[\label{fig: Imagen rad 1.5}]{
				\fbox{ \includegraphics[scale=1]{img/transformaciones/rotation1,5.png} }
			}
			\caption[Rotación de un caracter]{Rotación de un caracter. (a) Imagen original. (b) Imagen rotada 0.5 radianes. (c) Imagen rotada 1 radian. (d) Imagen rotada 1.5 radianes}
			\label{fig: Transformacion Afin - Rotacion}
		\end{figure}	
			
		\paragraph{Escala}
			
			La escala es una transformación que busca modificar el tamaño del objeto relativo a la ventana. Formalmente, se puede representar por la siguiente configuración de la matriz $M$:
			\begin{equation}
				M = 
				\begin{bmatrix}
					a & 0 \\
					0 & d \\
				\end{bmatrix}
			\end{equation}
		donde $a$ y $d$ representan los cambios en los ejes $x$ e $y$ de la imagen respectivamente. Se puede apreciar la aplicación de esta transformación en la Figura \ref{fig: Transformacion Afin - Escala}.
		\begin{figure}[htbp]
			\centering
			\subfloat[\label{fig: escala original}]{
				\fbox{ \includegraphics[scale=1]{img/transformaciones/original.png} }
			}
			\subfloat[\label{fig: Imagen escala 0.8}]{
				\fbox{ \includegraphics[scale=1]{img/transformaciones/scale0,8.png} }
			}
			\subfloat[\label{fig: Imagen escala 1.2}]{
				\fbox{ \includegraphics[scale=1]{img/transformaciones/scale1,2.png} }
			}
			\caption[Cambio de escala de un carácter]{Cambio de escala de un carácter. (a) Imagen original. (b) Imagen ampliada $1.2$ veces su tamaño  (c) Imagen reducida a $0.8$ veces su tamaño .}
			\label{fig: Transformacion Afin - Escala}
		\end{figure}	
			
		\paragraph{Transvección}
		
			La transvección es una transformación donde la imagen se rota en un solo eje. Es una función que toma un punto con coordenadas $(x,y)$ al punto $(x +ny, y)$, donde $n$ es un parámetro fijo, denominado el factor de inclinación. El efecto es el desplazamiento de todos los puntos horizontalmente en una cantidad proporcional a su coordenada $y$. Todo punto encima del eje $x$ es desplazado a la derecha si $n > 0$, y a la izquierda si $n < 0$. La matriz $M$ va a ser de la forma:
			\begin{equation}
				M = 
				\begin{bmatrix}
					1 & n \\
					0 & 1 \\
				\end{bmatrix}.
			\end{equation}
		Se puede observar en la Figura \ref{fig: Transformacion Afin - Transveccion} los diferentes efectos de aplicar diferentes valores de $n$ en la aplicación de la transvección en una imagen.
		\begin{figure}[htbp]
			\centering
			\subfloat[\label{fig: Imagen transveccion 0.5}]{
				\fbox{ \includegraphics[scale=1]{img/transformaciones/shear0,5.png} }
			}
			\subfloat[\label{fig: Imagen transveccion 0.2}]{
				\fbox{ \includegraphics[scale=1]{img/transformaciones/shear0,20.png} }
			}
			\subfloat[\label{fig: Imagen original}]{
				\fbox{ \includegraphics[scale=1]{img/transformaciones/original.png} }
			}
			\subfloat[\label{fig: Imagen transveccion -0.2}]{
				\fbox{ \includegraphics[scale=1]{img/transformaciones/shear-0,20.png} }
			}
			\subfloat[\label{fig: Imagen transveccion -0.5}]{
				\fbox{ \includegraphics[scale=1]{img/transformaciones/shear-0,5.png} }
			}
			\caption[Transvección de un carácter]{Transvección de un carácter.}
			\label{fig: Transformacion Afin - Transveccion}
		\end{figure}	
		
		\paragraph{Traslación}		
			
			La traslación es el último tipo de transformación afín que se usa en este trabajo y a diferencia del resto de las transformaciones, esta no hace uso de $M$, sino de $b$ que es el vector de desplazamiento. Formalmente:
			\begin{equation*}
				M =  
					\begin{bmatrix}
						1 & 0 \\
						0 & 1  \\
					\end{bmatrix}
					b =
					\begin{bmatrix}
						b_1 \\
						b_2 \\
					\end{bmatrix}	
			\end{equation*}
		donde $b_1$ y $b_2$ representan el movimiento en el eje $x$ e $y$ respectivamente.
		
	El siguiente tipo de transformaciones son las \textit{fotométricas}. Este tipo de transformaciones se caracterizan por surgir de variaciones en la iluminación de una imagen, es decir, es una variación en la función de intensidad de la misma. En general, este tipo de transformaciones no cambian la forma de la imagen. Por ejemplo, el ruido Gaussiano se origina, entre otras cosas, por la pobre iluminación del entorno, en el caso de las imágenes digitales. Dentro de este grupo podemos encontrar las siguientes transformaciones:
		
		\paragraph{Suavizado Gaussiano}
		
			Es una técnica que consta de la difuminación de una imagen a partir de la convolución con una función Gaussiana. Se usa principalmente para reducir el ruido en una imagen simulando el efecto de ``blur''. Dadas las coordenadas de un punto $(x, y)$ en una imagen, la función de suavizado es la siguiente:
			\begin{align*}
				G(x,y) = \frac{1}{2\pi\sigma^2}\textit{e}^{-\frac{x^2+y^2}{2\sigma^2}}
			\end{align*}
			donde $\sigma$ es la desviación estándard de la distribución gaussiana. En la figura \ref{fig: Suavizado Gaussiano}, se puede observar el efecto para valores crecientes de $\sigma$.
			
		\begin{figure}[htbp]
			\centering
			\subfloat[\label{fig: Imagen original sin blur	}]{
				\fbox{ \includegraphics[scale=1]{img/transformaciones/original.png} }
			}
			\subfloat[\label{fig: Imagen con blur 1}]{
				\fbox{ \includegraphics[scale=1]{img/transformaciones/blur1.png} }
			}
			\subfloat[\label{fig: Imagen con blur 2}]{
				\fbox{ \includegraphics[scale=1]{img/transformaciones/blur2.png} }
			}
			\caption[Suavizado Gaussiano de un carácter]{Aplicación de blur o suavizado gaussiano a un carácter. (a) Imagen original. (b) Imagen suavizada con un valor de $\sigma = 1$.  (c) Imagen aún más suavizada con un valor de $\sigma = 2$ .}
			\label{fig: Suavizado Gaussiano}
		\end{figure}				
			
			
		\paragraph{Ruido Gaussiano}			
			
			El ruido en una imagen es una variación en la información sobre el color o la iluminación en la misma. Puede ser producido por un sensor, la circuitería de un escáner o una cámara digital. Condiciones de poca iluminación o interferencia electromagnética durante la transmisión de las imágenes son factores para que aparezcan este tipo de distorsiones. El ruido Gaussiano en las imágenes es un tipo de ruido que se caracteriza por tener una distribución Gaussiana. La figura \ref{fig: Ruido Gaussiano} presenta 2 ejemplos de imágenes de caracteres con esta transformación. A medida que el parámetro $\sigma$  aumenta, el ruido en la imagen se hace más notorio.
			
			Formalmente se puede expresar esta transformación de la siguiente manera:
			
			$$x \rightarrow x + \epsilon $$
			
			donde $\epsilon$ es una muestra asociada a una variable aleatoria con distribución $\mathcal{N}(0, \sigma^2)$
			
		\begin{figure}[htbp]
			\centering
			\subfloat[\label{fig: Imagen original sin ruido	}]{
				\fbox{ \includegraphics[scale=1]{img/transformaciones/original.png} }
			}
			\subfloat[\label{fig: Imagen con ruido 15}]{
				\fbox{ \includegraphics[scale=1]{img/transformaciones/noise15.png} }
			}
			\subfloat[\label{fig: Imagen con ruido 30}]{
				\fbox{ \includegraphics[scale=1]{img/transformaciones/noise30.png} }
			}
			\caption[Ruido Gaussiano en un carácter]{Aplicación de ruido gaussiano a un carácter. (a) Imagen original. (b) sigma=15.  (c) sigma=30 .}
			\label{fig: Ruido Gaussiano}
		\end{figure}
			
	%Dado que HOG requiere imágenes en escala de grises para poder extraer las características, en este trabajo sólo se consideran imágenes con esta condición.
			
			%\paragraph{Anexar caracteres}
					
			%Otro tipo de modificación que vale la pena destacar, es la de anexar caracteres a los costados de la imagen a alterar. Esto es debido a que la mayoría de las imágenes de caracteres reales son extraídas de entornos donde la misma forma parte de una palabra. Es por eso que en la mayoría de los conjuntos generados, cada carácter está acompañado de pedazos de otros caracteres.
			
		%\begin{figure}[htbp]
		%	\centering
		%	\subfloat[\label{fig: Imagen anexo 1	}]{
		%		\fbox{ \includegraphics[scale=1]{img/transformaciones/anexo1.png} }
		%	}
		%	\subfloat[\label{fig: Imagen anexo 2}]{
		%		\fbox{ \includegraphics[scale=1]{img/transformaciones/anexo2.png} }
		%	}
		%	\subfloat[\label{fig: Imagen anexo 3}]{
		%		\fbox{ \includegraphics[scale=1]{img/transformaciones/anexo3.png} }
		%	}
		%	\caption[Caracteres pegados]{Imágenes de caracteres con caracteres pegados a izquierda y derecha.}
		%	\label{fig: Imagen anexos}
		%\end{figure}

		%	La creación de estos datos sintéticos son para el conjunto de entrenamiento del clasificador. El conjunto de test esta conformado por 15 imágenes reales por clase obtenidas del dataset \textit{Chars74K-15}, a las cuales no se les modifico en absoluto.

		\paragraph{Extracción de características y binarización} ~\\
		
			La siguiente etapa en el pipeline es la extracción del vector de características (ver sección ``Vectores de características'' \ref{subsection:feature}), de cada imagen del conjunto de entrenamiento y la binarización posterior de dicho descriptor (ver sección ``Binarización'' \ref{subsubsection:binarizacion}).
			
			Una de las condiciones para poder extraer el vector de características de una imagen con HOG, es que esta esté en escala de grises, requisito para el resto del trabajo. Luego, dada una imagen, ventana que contiene un carácter, se extrae el descriptor de características a través de HOG y se obtiene un vector $v \in \mathbb{R}^{K}$, donde $K$ depende de los parámetros del algoritmo. Posteriormente se procede como se detalla en \ref{subsubsection:binarizacion} donde se obtiene el umbral que después se utiliza para binarizar a $v$.
			
			%Dado un conjunto de entrenamiento con $J$ imágenes, se construye una matriz $J \times K$ donde $K$ es la dimensión del descriptor HOG para cada imagen. Cabe aclarar que todas las imágenes tienen que tener las mismas dimensiones (48x48 píxeles) de lo contrario no se podría construir la matriz. Esto se debe a que HOG devuelve en sí un vector $v \in \mathbb{R}^{K}$ (donde $K$ depende de los parámetros que se le pasen al algoritmo y a la dimensión de la imagen). Posteriormente se procede como se detalla en \ref{subsection:hog} donde se obtiene el umbral que después se utiliza para binarizar todos los descriptores tanto del conjunto de entrenamiento como del conjunto de evaluación.
			Supongamos un conjunto de entrenamiento con $N$ imágenes. Luego, para generar el vector umbral en los experimentos, se propuso la siguiente implementación basada en la detallada en \ref{subsubsection:binarizacion}.
			
			\begin{itemize}
				\item Dado $N$ descriptores HOG de dimensión $D$, se forma una matriz de tamaño $N \times D$ donde cada fila representa un vector.
				\item Se seleccionan $X$ columnas al azar de la matriz, con reemplazo.
				\item Se aplica una función de umbralización sobre cada columna, la función es una de las detalladas en \ref{subsubsection:binarizacion}. Se obtiene de esta manera un vector nuevo $W$ de dimensión $X$ tal que cada elemento de $W$ está compuesta por un par $(z,y)$. $y$ es un número talque $0 \leq y \leq D$ que representa una de las columnas elegidas de la matriz y $z$ representa el valor resultante de haber aplicado la función elegida a dicha columna. Cabe aclarar que $X$ puede ser mayor o menor a $D$ por lo cual el umbral $W$ puede tener mayor o menor dimensión al final.
				\item Posteriormente dicho umbral $W$ se utiliza para binarizar los vectores originales de manera sencilla: sea $v_j$, con $j \in \{1,\dots,N\}$, uno de los $N$ vectores originales y tal que $v_j = d_1,d_2,\dots,d_D$. Luego se compara cada entrada del umbral $W$ con la $y$-esima entrada del vector $v_j$. Si $d_y \leq z$ se asigna 0, caso contrario 1. De esta manera binarizamos el vector $v_j$ obteniendo un nuevo vector binario de dimensión $X$.
			
			\end{itemize}

		\paragraph{Entrenamiento del clasificador}  ~\\

			En esta etapa, al igual que los autores del trabajo original, se utiliza como clasificador a Random Ferns. Para poder entrenar al clasificador, los vectores modificados de la etapa anterior se almacenan en un diccionario. Este último, esta estructurado como un conjunto de diccionarios anidados y representa la ``base de datos'' del sistema.

			Dado que los vectores binarizados pertenecen al conjunto $\{ 0,1 \}^{N}$, si los quisiéramos almacenar en una sola tabla, habría una tabla por clase, cada una debería tener $2^{N}$ entradas lo cual si $N$ es muy grande sería imposible por la cantidad de memoria necesitada para mantener estas tablas en memoria. Luego, basándonos en lo explicado sobre Random Fern en \ref{subsection:ferns}, la solución pasa por dividir cada vector en $M$ grupos de longitud $S = \frac{N}{M}$. Con esto, un vector completo esta almacenado en $M$ tablas diferentes, tablas correspondiente a la clase del vector. En total estaríamos trabajando con $M \times 62$ tablas.

			Dado un vector $v=(v_1, \dots, v_S)$  tal que el mismo está dividido en $M$ grupos de longitud $S$, es decir, $v=(w_1,\dots,w_M)$, donde $w_i=(w_i,\dots,w_{i+S})$. Sea $z$ la clase de $v$ y sean, $t_i~ i=1, \dots, S$, las tablas de $z$. Cada grupo $w_i$, $i=1, \dots, M$, va a tener una entrada en su tabla correspondiente $z_i$ equivalente a su representación decimal. Dicha entrada en la tabla tiene un contador que se va incrementando en $1$ a medida que se acceda a esta entrada. Luego, a medida que se van ``entrenando'' las tablas de cada clase, se puede dar la situación, más a medida que la dimensión de cada grupo crece, que algunas entrada no se hayan accedido nunca. Esto supone un problema para los calculos posteriores del clasificador por lo que se propone inicializar cada entrada de cada tabla con un valor $\alpha \neq 0$. 

		\paragraph{Evaluación del clasificador} ~\\

			Para la evaluación del clasificador, se toma como conjunto de test el descrito en la sección de descripción del dataset. A cada imagen del conjunto de test, se le extrae el descriptor HOG, se lo binariza y con el vector resultante se calcula la probabilidad de que dicho vector pertenezca a cada una de las clases involucradas. El calculo para cada clase es el mismo y consiste en dividir el vector de prueba en $M$ grupos y por cada grupo obtener el valor en la tabla correspondiente. Por cada clase entonces, tendríamos $M$ valores. Finalmente, se realiza la suma de los logaritmos de cada valor y el resultado es la probabilidad de que ese vector pertenezca a la clase evaluada. Al final se le asigna a la imagen de prueba la clase que haya obtenido la mayor probabilidad.

