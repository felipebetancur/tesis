\newpage	
\section{Reconocimiento de texto en escenas naturales}

	En el capítulo anterior, se desarrollaron los conceptos necesarios para entender las bases del aprendizaje supervisado y los clasificadores probabilísticos. Todos estos conceptos sirven para entender los principios sobre los que se asienta el trabajo realizado por Wang et. al. en \cite{wang}. El trabajo de los autores, abarca el reconocimiento de texto en escenas naturales en todas sus etapas. En particular, una de ellas es el reconocimiento de caracteres. Esta etapa, plantea un problema de clasificación (multiples clases representadas por los caracteres) que requiere de un clasificador que pueda manipular varias clases de manera eficiente. De ahí surge la necesidad de usar Random Ferns. La precisión y eficiencia de este último lo convierten en la herramienta ideal para afrontar este problema.
	
	En este capitulo se van a presentar dos implementaciones: la provista originalmente por Wang et al. en \cite{wang} y la realizada en este trabajo basándose en la primera. Solamente se van a desarrollar los temas referentes al reconocimiento de caracteres ya que el presente trabajo se enfoca únicamente en ese problema. Como complemento a ambas implementaciones, hay una sección dedicada al papel que juegan las imágenes en ambos trabajos. Se explica la importancia de las mismas y cómo se extraen los descriptores de las mismas.
	
	
	\subsection{Introducción}

	En el trabajo que realizaron Wang et al., los autores identificador el problema que había al detectar y reconocer palabras en imágenes naturales. Ellos identifican, que si bien las actuales aplicaciones de OCR se manejan bien con documentos escaneados, el texto adquirido en entornos naturales (también referido como texto de escena) se ha vuelto más frecuente. Todo esto debido al aumento de dispositvos que son capaces de extraer dicha información, sean estos celulares, tabletas o cámaras.
	
	Con la salida del primer dataset público denominado \textit{ICDAR}, el cual resaltaba el problema de detectar y reconocer texto de escena, los organizadores del mismo, identificaron cuatro subproblemas.
	\begin{itemize}
		\item La clasificación de caracteres recortados.
		\item Detección de texto en la imagen completa.
		\item El reconocimiento de palabras recortadas.
		\item El reconocimiento de palabras en la imagen completa.
	\end{itemize}
	
	Dada esta problemática, los autores se enfocaron en el problema del reconocimiento de texto. Para poder encarar esto, incorporaron una lista de palabras (i.e., un lexicón) para detectar y leer.
		
	Para esto, ellos construyen y evaluan dos sistemas. El primero consiste en un pipeline de dos etapas que consiste en la detección de texto seguido del reconocimiento a partir de un destacado motor de OCR. El segundo, es un sistema arraigado en el reconocimiento de objetos genéricos, el cual es una extensión de un trabajo que realizaron anteriormente \cite{WB10}.
	
	Sus contribuciones son las siguiente:
		\begin{itemize}
			\item Evaluan la performance en la detección y el reconocimiento de palabras de un enfoque de dos etapas que consiste en un detector de texto (que es estado del arte) y un destacado motor de OCR.
			\item Construyen un sistema basado en su trabajo anterior \cite{WB10} y muestran que sus pipelines de reconocimiento realizan una mejor tarea a comparación al pipeline convencional de OCR.
		\end{itemize}	
	
	\subsection{Reconocimiento de caracteres}

	\begin{itemize}
		\item Algoritmos usados
			\begin{itemize}
				\item Introducción a Random Ferns: explicar el algoritmo y porqué lo usaron.
				\item HOG: hacer una ligera mensión de su uso. Los detalles van a estar en la sección correspondiente en el capítulo 2.
				\item non-maximal suppression (NMS)
			\end{itemize}
		\item Datasets usados
			\begin{itemize}
				\item Aca puede ir una explicación del uso de datos sintéticos.			
			\end{itemize}			 
	\end{itemize}
	
	\RC{Abajo, borrador}
	
	El reconocimiento de caracteres, es la primera etapa en el pipeline de procesamiento que desarrollaron Wang et al. Para esto, realizaron una detección a múltiple escala usando un algoritmo de clasificación de ventana deslizante. Dado que la cantidad de clases a detectar eran muchas (62 clases), ellos decidieron usar \textit{random ferns} como su clasificador. Esto es debido a que es un clasificador multi-clase y es eficiente.
	
	Random ferns fue explicado con anterioridad en la sección \ref{subsection:ferns}. Para la obtención de las características, los autores hacen uso de los descriptores HOG  los cuales binarizan (ver \ref{subsection:hog}) aplicando umbrales aleatorios con el objetivo de obtener vectores de características que puedan ser fácilmente almacenados y accesibles a través de una tabla. Como última etapa en el reconocimiento de caracteres, realizan \textit{non-maximal suppression} sobre cada carácter usando la siguiente heurística: iteran sobre todas las ventanas en la imagen en orden descendiente de su puntaje, si la ubicación no fue suprimida, suprimen todos sus vecinos.
	
	Uno de los problemas al entrenar un clasificador de caracteres, es el de encontrar un dataset de entrenamiento lo suficientemente grande para obtener buenos resultados. Una forma de solventar este problema, es generar un dataset con imágenes sintéticas, ya que, además de la  obvia ventaja de tener una cantidad ilimitada de datos, permite tener control sobre las dimensiones de cada imagen. Este enfoque fue aprovechado por Wang et al., que sintetizaron alrededor de 1000 imágenes por carácter usando 40 fuentes. A cada imagen, le agregaban una cierta cantidad de ruido gaussiano y le aplicaban transformaciones afines aleatorias. Con esto, obtenían imágenes de caracteres que intentaban asemejarse a las reales.
	
	
	
	%\subsection{Reconocimiento de palabras}
	\begin{itemize}
		\item Introducción a Pictorial Structures (Teoría)
		\item Uso de PS + Lexicón
		\begin{itemize}
			\item Algoritmo PLEX
		\end{itemize}
		\item Re-scoring y NMS
		\begin{itemize}
			\item Problemas con PLEX
		\end{itemize}
		\item Implementación
	\end{itemize}
	
	\subsection{Arquitectura del sistema}
\label{subsection:impl_propia}

	El presente trabajo, como se dijo con anterioridad, sienta sus bases en el trabajo realizado por Wang et al. salvo que el enfoque del mismo se centra en los problemas de reconocmiento de caracteres y el reconocimiento de palabras. En las próximas subsecciones, se procederá a explicar cuestiones de implementación que se tuvieron en cuenta al momento de resolver ambos problemas.

	\subsubsection{Reconocimiento de caracteres}
	\label{subsubsection:recon-caracteres}
		\paragraph{Pipeline de procesamiento} ~\\

			La implementación que se realizó en este trabajo, está basada en un pipeline similar al que realizaron Wang et al. que comprende dos instancias: la instancia de entrenamiento y la instancia de evaluación. La primera esta constituida por la generación de datos sintéticos que constituyen el conjunto de entrenamiento, la extracción de las características, la binarización y el entrenamiento del clasificador. La segunda instancia consiste en la evaluación del clasificador que abarca desde la extracción y binarización de las características del conjunto de prueba y posteriormente la evaluación del clasificador para cada una de estas. El pipeline se puede apreciar en la siguiente imagen:

			\begin{figure}[htbp]
				\centering
				\fbox{ \includegraphics[scale=0.5]{img/OCR_pipeline_1.jpg} }
				\caption[Pipeline de procesamiento]{Esto es un ejemplo de como quedaría visualmente pero no es el pipeline de este trabajo.}
				\label{fig: Pipeline de mi sistema}
			\end{figure}

		\paragraph{Generación de datos sintéticos} ~\\

			La datos sintéticos que se generan y utilizan en este trabajo son extraidos del dataset \textit{Chars74K} el cual está compuesto, entre otras cosas, por 62992 imágenes de caracteres sintéticos extraidos de fuentes de computadora. Cada una de las clases involucradas en la clasificación tienen un poco más de 1000 de estas imágenes (son 62 clases en total). El objetivo detrás de la generación de estos datos sintéticos es agregar vistas de los caracteres que no están reflejadas en el dataset original pero que son plausibles de ser observadas en la realidad.
			
			Inicialmente, se tiene como base un conjunto el cual contiene imágenes de caracteres de diversas fuentes. Lo que se busca, es aplicar a cada imagen un conjunto de transformaciones afines aleatorias con el objetivo final de obtener una imagen nueva con la apariencia  lo más cercana a una real. Una transformación afin representa en esencia una relación entre dos imágenes. La misma se aplica entre dos espacios afines (son estructuras geométricas que generalizan las propiedades afines del espacio Euclideo) y consiste en una transformación lineal seguida de una traslación. Formalmente, es una función entre espacio afines . Si $X$ e $Y$ son espacios afines, luego cada transformación afín $f:X \rightarrow Y$ es de la forma $x\mapsto Mx + b$, donde $M$ es una transformación lineal en $X$ y $b$ es un vector en $Y$. Se usan las multiplicaciones entre matrices para representar las transformaciones lineales y la suma de vectores para representar las traslaciones. Mediante matrices ampliadas, sin embargo, es posible representar ambos tipos de transformaciones.	La matriz $M$ presentada anteriormente es una matriz $2\times 2$ y permite realizar transformaciones en imágenes de 2 dimensiones. En este trabajo, para generar los datos sintéticos se hacen uso de las siguientes transformaciones:
			
			\paragraph{Rotación}
			
				La rotación es una transformación afín que consta en rotar el ángulo de la imágen en el sentido anti-horario. El parámetro usado para el mismo son los radianes y las diferentes variaciones se pueden apreciar en la figura \ref{fig: Transformacion Afin - Rotacion}. Teniendo en cuenta la definición formal de una transformación afín anterior, la rotación formalmente se puede representar de la siguiente manera:

			\begin{equation*}
					M =  
					\begin{bmatrix}
						cos(\theta) & -sin(\theta) \\
						sin(\theta) & cos(\theta)  \\
					\end{bmatrix}
					b =
					\begin{bmatrix}
						0 \\
						0 \\
					\end{bmatrix}	
			\end{equation*}

	donde se puede observar el parámetros $\theta$ que representa el grado de inclinación y el vector $b$ es nulo por lo cual no hay traslación alguna.

		\begin{figure}[htbp]
			\centering
			\subfloat[\label{fig: sintetica original}]{
				\fbox{ \includegraphics[scale=1]{img/transformaciones/original.png} }
			}
			\subfloat[\label{fig: Imagen rad 0.5}]{
				\fbox{ \includegraphics[scale=1]{img/transformaciones/rotation0,5.png} }
			}
			\subfloat[\label{fig: Imagen rad 1}]{
				\fbox{ \includegraphics[scale=1]{img/transformaciones/rotation1.png} }
			}
			\subfloat[\label{fig: Imagen rad 1.5}]{
				\fbox{ \includegraphics[scale=1]{img/transformaciones/rotation1,5.png} }
			}
			\caption[Rotación de un caracter]{Rotación de un caracter. (a) Imagen original. (b) Imagen rotada 0.5 radianes. (c) Imagen rotada 1 radian. (d) Imagen rotada 1.5 radianes}
			\label{fig: Transformacion Afin - Rotacion}
		\end{figure}	
			
		\paragraph{Escala}
			
			La escala es una transformación que busca modificar el tamaño del objeto al cual se la aplica haciendola más chica o más grande lo que era originalmente. Formalmente, se puede representar por la siguiente configuración de la matriz $M$:
			\begin{equation}
				M = 
				\begin{bmatrix}
					a & 0 \\
					0 & d \\
				\end{bmatrix}
			\end{equation}
		donde $a$ y $d$ representan los cambios en los ejes $x$ e $y$ de la imagen respectivamente. Se puede apreciar la aplicación de esta transformación en la figura \ref{fig: Transformacion Afin - Escala}.
		\begin{figure}[htbp]
			\centering
			\subfloat[\label{fig: escala original}]{
				\fbox{ \includegraphics[scale=1]{img/transformaciones/original.png} }
			}
			\subfloat[\label{fig: Imagen escala 0.8}]{
				\fbox{ \includegraphics[scale=1]{img/transformaciones/scale0,8.png} }
			}
			\subfloat[\label{fig: Imagen escala 1.2}]{
				\fbox{ \includegraphics[scale=1]{img/transformaciones/scale1,2.png} }
			}
			\caption[Cambio de escala de un caracter]{Cambio de escala de un caracter. (a) Imagen original. (b) Imagen ampliada $1.2$ veces su tamaño  (c) Imagen reducida a $0.8$ veces su tamaño .}
			\label{fig: Transformacion Afin - Escala}
		\end{figure}	
			
		\paragraph{Transvección}
		
			La transvección es una transformación donde la imagen se rota en un solo eje. Es una función que toma un punto con coordenadas $(x,y)$ al punto $(x +ny, y)$ donde $n$ es un parámetro fijo, denominado el factor de inclinación. El efecto es el desplazamiento de todos los puntos horizontalmente en una cantidad proporcional a su coordenada $y$. Todo punto encima del eje $x$ es desplazado a la derecha si $n > 0$, y a la izquiera si $n < 0$. Nuestra matriz $M$ va a ser de la forma:
			\begin{equation}
				M = 
				\begin{bmatrix}
					1 & n \\
					0 & 1 \\
				\end{bmatrix}.
			\end{equation}
		Se puede observar en la figura \ref{fig: Transformacion Afin - Transveccion} los diferentes efectos de aplicar diferentes valores de $n$ en la aplicación de la transvección en una imagen.
		\begin{figure}[htbp]
			\centering
			\subfloat[\label{fig: Imagen transveccion 0.5}]{
				\fbox{ \includegraphics[scale=1]{img/transformaciones/shear0,5.png} }
			}
			\subfloat[\label{fig: Imagen transveccion 0.2}]{
				\fbox{ \includegraphics[scale=1]{img/transformaciones/shear0,20.png} }
			}
			\subfloat[\label{fig: Imagen original}]{
				\fbox{ \includegraphics[scale=1]{img/transformaciones/original.png} }
			}
			\subfloat[\label{fig: Imagen transveccion -0.2}]{
				\fbox{ \includegraphics[scale=1]{img/transformaciones/shear-0,20.png} }
			}
			\subfloat[\label{fig: Imagen transveccion -0.5}]{
				\fbox{ \includegraphics[scale=1]{img/transformaciones/shear-0,5.png} }
			}
			\caption[Transvección de un caracter]{Cambio de transvección de un carácter.}
			\label{fig: Transformacion Afin - Transveccion}
		\end{figure}	
		
		\paragraph{Traslación}		
			
			La traslación es el último tipo de transformación afín que usamos en este trabajo y a diferencia del resto de las transformaciones, esta no hace uso de $M$, sino de $b$ que es el vector de desplazamiento. Formalmente:
			\begin{equation*}
				M =  
					\begin{bmatrix}
						1 & 0 \\
						0 & 1  \\
					\end{bmatrix}
					b =
					\begin{bmatrix}
						b_1 \\
						b_2 \\
					\end{bmatrix}	
			\end{equation*}
		donde $b_1$ y $b_2$ representan el movimiento en el eje $x$ e $y$ respectivamente. Particularmente no usamos esta transformación para mover los caracteres, sino más bien para mantener la imagen del carácter centrada durante las diferentes transformaciones que le aplicamos a la imagen.
		
		
		\paragraph{Suavizado Gaussiano}
		
			Es una técnica que consta de la difuminación de una imagen a partir de la aplicación de una función gaussiana. Se usa principalmente para reducir el ruido en una imagen. Dados das las coordenadas de un punto $(x, y)$ en una imagen, la función de suavizado es la siguiente
			\begin{align*}
				G(x,y) = \frac{1}{2\pi\sigma^2}\epsilon^{-\frac{x^2+y^2}{2\sigma^2}}
			\end{align*}
			donde $\sigma$ es la desviación estandard de la distribución gaussiana. En la figura \ref{fig: Suavizado Gaussiano}, se puede observar como distintos valores para $\sigma$ a medida que aumenta va distorcionando la imagen un poco más.
			
		\begin{figure}[htbp]
			\centering
			\subfloat[\label{fig: Imagen original sin blur	}]{
				\fbox{ \includegraphics[scale=1]{img/transformaciones/original.png} }
			}
			\subfloat[\label{fig: Imagen con blur 1}]{
				\fbox{ \includegraphics[scale=1]{img/transformaciones/blur1.png} }
			}
			\subfloat[\label{fig: Imagen con blur 2}]{
				\fbox{ \includegraphics[scale=1]{img/transformaciones/blur2.png} }
			}
			\caption[Suavizado Gaussiano de un caracter]{Aplicación de blur o suavizado gaussiano a un caracter. (a) Imagen original. (b) Imagen suavizada con un valor de $\sigma = 1$  (c) Imagen aún más suavizada con un valor de $\sigma = 2$ .}
			\label{fig: Suavizado Gaussiano}
		\end{figure}				
			
			
		\paragraph{Ruido Gaussiano}			
			
			El ruido en una imagen es una variación en la información sobre el color o la iluminación en la misma. Puede ser producido por un sensor, la circuitería de una escaner o una cámara digital. Condiciones de poca iluminación o interferencia electromagnética durante la transmisión de las imágenes son factores para que aparezca este tipo de distorsiones. El ruido gaussiano en las imágenes es un tipo de ruido que se caracteríza por tener una distribución gaussiana. La figura \ref{fig: Ruido Gaussiano} presenta 2 ejemplos de imágenes de caracteres con ruido gaussiano. A medida que el parámetro $\sigma$  aumenta, el ruido en la imagen se hace más notorio.
			
		\begin{figure}[htbp]
			\centering
			\subfloat[\label{fig: Imagen original sin ruido	}]{
				\fbox{ \includegraphics[scale=1]{img/transformaciones/original.png} }
			}
			\subfloat[\label{fig: Imagen con ruido 15}]{
				\fbox{ \includegraphics[scale=1]{img/transformaciones/noise15.png} }
			}
			\subfloat[\label{fig: Imagen con ruido 30}]{
				\fbox{ \includegraphics[scale=1]{img/transformaciones/noise30.png} }
			}
			\caption[Ruido Gaussiano en un caracter]{Aplicación de ruido gaussiano a un caracter. (a) Imagen original. (b) sigma=15  (c) sigma=30 .}
			\label{fig: Ruido Gaussiano}
		\end{figure}
			
			Un punto a destacar en este proceso, es que las fuentes de letras están en una escala de grises y se mantienen de esta forma. Esto es debido a que, durante el entrenamiento, solo es posible obtener las características de las imágenes sí y sólo sí estas están en escala de grises (requerimiento del algoritmo HOG). 
			
		\paragraph{Anexar caracteres}
					
			Otro tipo de modificación que vale la pena destacar, es la de anexar caracteres a los costados de la imagen a alterar. Esto es debido a que la mayoría de las imágenes de caracteres reales son extraidas de entornos donde la misma forma parte de una palabra. Es por eso que en la mayoría de los conjuntos generados, cada caracter está acompañado de pedazos de otros caracteres.
			
		\begin{figure}[htbp]
			\centering
			\subfloat[\label{fig: Imagen anexo 1	}]{
				\fbox{ \includegraphics[scale=1]{img/transformaciones/anexo1.png} }
			}
			\subfloat[\label{fig: Imagen anexo 2}]{
				\fbox{ \includegraphics[scale=1]{img/transformaciones/anexo2.png} }
			}
			\subfloat[\label{fig: Imagen anexo 3}]{
				\fbox{ \includegraphics[scale=1]{img/transformaciones/anexo3.png} }
			}
			\caption[Caracteres pegados]{Imágenes de caracteres con caracteres pegados a izquierda y derecha.}
			\label{fig: Imagen anexos}
		\end{figure}

			La creación de estos datos sintéticos son para el conjunto de entrenamiento del clasificador. El conjunto de test esta conformado por 15 imágenes reales por clase obtenidas del dataset \textit{Chars74K-15}, a las cuales no se les modifico en absoluto. Esto es debido a que los experimentos de evaluación del clasificador que se realizen tienen que poder ser comparados con los de Wang et al. en iguales condiciones.


		\paragraph{Entrenamiento del clasificador}  ~\\

			La segunda etapa del pipeline corresponde al entrenamiento del clasificador. Al igual que los autores del trabajo original, en este se utiliza como clasificador a Random Ferns. Para poder entrenar al clasificador, se procede a extraer las características de las imágenes del conjunto de entrenamiento a través del algoritmo HOG. HOG devuelve en sí un vector $v \in \mathbb{R}^{n}$ (donde $n$ depende de los parámetros que se le pasen al algoritmo, explicado en el próximo capítulo) el cual hay que binarizar para poder almacenarlo.

			Para poder binarizar los vectores calculados, como se explicó en la sección \ref{subsection:hog}, se necesita de un umbral cuya dimensión es igual a la de los vectores. Dicho umbral binariza a un vector comparando sus valores dimensión a dimensión. Un vez realizado este procedimiento, los vectores modificados se almacenan en un diccionario. Este último, esta estructurado como un conjunto de diccionarios anidados y representa la base ``base de datos'' del sistema. Los detalles del porqué se eligió esta forma de almacenar la información se puede encontrar en el apéndice A. \RC{Crear apendice A donde se van a discutir cosas como lenguaje usado, base de datos, librerias entre otros}

			Dado que los vectores pertenecen al conjunto $\{ 0,1 \}^{N}$ si los quisiéramos almacenar en una sola tabla (habría una tabla por clase), cada una debería tener $2^{N}$ entradas lo cual si $N$ es muy grande sería imposible por la cantidad de memoria necesitada para mantener estas tablas en memoria. Luego, basándonos en lo explicado sobre Random Fern en \cite{subsection:ferns}, la solución pasa por dividir cada vector en $M$ grupos de dimensión $S = \frac{N}{M}$. Con esto, un vector completo esta almacenado en $M$ tablas diferentes (tablas correspondiente a la clase del vector). En total estaríamos trabajando con $M \times 62$ tablas (pues hay 62 clases diferentes).

			Dado un vector $v=(v_1, \dots, v_s)$ donde $v_i \in \{ 0,1 \}^{M}$, $z$ sea la clase de $v$ y sean $t_i i=1, \dots, s$ las tablas de $z$. Cada grupo $v_i i=1, \dots, s$ va a tener una entrada en su tabla correspondiente $z_i$ equivalente a su representación decimal. Inicialmente cada tabla está inicializada dado un parámetro $\alpha \neq 0$ que se explicará en detalle en el próximo capítulo.

		\paragraph{Evaluación del clasificador} ~\\

			Para la evaluación del clasificador, se toma como conjunto de test el descripto en la sección de descripción del dataset. A cada imágen del conjunto de test, se le extrae el descriptor HOG, se lo binariza y con el vector resultante se calcula la probabilidad de que dicho vector perteneza a cada una de las clases involucradas. El calculo para cada clase es el mismo y consiste en dividir el vector de prueba en $M$ grupos y por cada grupo obtener el valor en la tabla correspondiente. Por cada clase entonces tendríamos $M$ valores. Finalmente se realiza la suma de los logaritmos de cada valor y el resultado es la probabilidad de que ese vector pertenzca a la clase evaluada. Como es claro, al final se le asigna a la imagen evaluada la clase que haya obtenido la mayor probabilidad.

	\subsubsection{Reconocimiento de palabras}

		\RC{Todavía no he implementado nada de esto así que queda pendiente...}

