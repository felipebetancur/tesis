\subsection{Reconocimiento de caracteres}
	
	El reconocimiento de caracteres, es la primera etapa en el pipeline de procesamiento que desarrollaron Wang et al. Dada una imagen, primero es necesario detectar las potenciales ubicaciones de los caracteres dentro de la misma. Para lograr esto, los autores realizaron una detección a múltiple escala usando un algoritmo de clasificación de ventana deslizante. El mismo recorre la imagen detectando potenciales caracteres dentro de sí y aumenta su escala con el objetivo de poder detectar caracteres más grandes. Al momento de detectar un posible carácter en una ventana, dada la gran cantidad de clases (62 caracteres en total) que están involucradas en el reconocimiento, los autores optaron por usar \textit{random ferns} (ver subsección \ref{subsection:ferns}) como su clasificador. Esto es debido a que es un clasificador eficiente y puede manejar múltiples clases.
	
	Para poder reconocer un símbolo es necesario obtener el vector de características del mismo (ver subsección \ref{subsection:feature}). Para lograr esto, Wang et. al. hacen uso de los descriptores HOG (ver \ref{subsection:hog}) los cuales han demostrado ser utiles en tareas de clasificación \cite{DT05}; sin embargo no es suficiente para poder clasificar al carácter ya que \textit{Random Ferns} hace uso de descriptores binarios para la clasificación pues son fáciles de computar, almacenar y escala con la cantidad de clases. Para poder binarizar estos vectores es indispensable el uso de algún método que genere un umbral necesario para la binarización. En su trabajo, especifican que los vectores binarios consisten en la aplicación de umbrales elegidos aleatoriamente sobre entradas elegidas al azar del vector HOG. Dado que no especifican ningún método en particular para la generación de los umbrales, es necesario analizar el código que han provisto para poder saberlo. Como última etapa en el reconocimiento de caracteres, realizan \textit{supresión de los máximos} (\textit{NMS} por sus siglas en inglés) sobre cada carácter. Es una técnica muy usada en los algoritmos de visión por computadora  y básicamente suprime los píxeles en la imagen que no son máximos locales (con respecto a los píxeles vecinos) a lo largo de la dirección del gradiente. Para cada carácter  usan la siguiente heurística: iteran sobre todas las ventanas en la imagen en orden descendiente de su puntaje, si la ubicación no fue suprimida, suprimen todos sus vecinos.
	
	Uno de los problemas al entrenar un clasificador de caracteres, es encontrar un conjunto de entrenamiento lo suficientemente grande para obtener buenos resultados. Una forma de solventar este problema, es generar un conjunto con imágenes sintéticas. Esto supone una gran ventaja por la cantidad ilimitada de datos que se pueden manejar. Dada la gran variabilidad de apariencias que hay en las imagenes reales y a lo difícil que es recolectar un conjunto de este tipo, es que surge esta alternativa. Este enfoque fue aprovechado por los autores, que sintetizaron alrededor de 1000 imágenes por carácter usando 40 fuentes. El objetivo de esto era obtener imágenes de caracteres que se asemejaran a las imágenes reales. Para lograr esto, a cada imagen de fuente le aplicaban una serie de transformaciones que alteraban su aspecto en un intento de imitar los diferentes aspectos que podía tener una imagen natural. En las próximas secciones se van a detallar algunas de estas transformaciones aplicadas a diferentes imágenes.
	

