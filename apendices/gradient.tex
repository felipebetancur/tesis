\appendix
\section{Gradiente}
\label{section:Apendice-Gradiente}

En este apéndice, se busca realizar una introducción básica al concepto de gradiente con el objetivo de poder comprender mejor la sección de HOG.

Sea $f(x_1,\dots,x_n)$ una función escalar de múltiples variables. Como expresa Gonzales et. al. en \cite{GonWoods}, el gradiente de $f$ es un vector que apunta en la dirección donde se registra la mayor tasa de incremento de la función. Su magnitud es la pendiente del gráfico en esa dirección. Es la generalización del concepto de derivada en funciones de múltiples variables.
		
	El gradiente de la función $f$ descrita anteriormente, es denotado como $\nabla f$ donde $\nabla$ (el símbolo nabla) denota el operador diferencial. El gradiente de $f$ es definido como el único campo vectorial cuyo producto punto con cualquier vector $v$ en cada punto $x$ es la derivada direccional de $f$ a lo largo de $v$. Es decir,
		 \begin{align*}
		 	(\nabla f(x))\cdot v = D_v f(x)
		 \end{align*}
		 
	En un sistema de coordenadas rectangular, el gradiente es el campo vectorial cuyos componentes son las derivadas parciales de $f$:
		 
		 \begin{align*}
		 	\nabla f(x) = \frac{\partial f}{\partial x_1}\mathbf{e}_1 + \cdots + \frac{\partial f}{\partial x_n }\mathbf{e}_n
		 \end{align*}
	donde los $\mathbf{e}_i$ son vectores unitarios ortogonales que apuntan en la dirección de coordenadas.

	En el procesamiento de imágenes, un gradiente es un cambio direccional en la intensidad o color de la imagen. En \cite{DJacobs}, Jacobs explica que el vector gradiente se forma combinando la derivada parcial de la imagen en las direcciones $x$ e $y$. Se puede expresar del a siguiente forma:
		\begin{align}
			\nabla I = \left( \frac{\partial I}{\partial x} , \frac{\partial I}{\partial y} \right)
		\end{align}	
		
	donde \textit{I}: $\mathbb{R}^{2} \rightarrow [0, 1]$ es la ``función intensidad'' que asigna un valor de intensidad a cada pixel (par (x,y)) de la imagen. Según Jacobs, cuando determinamos la derivada parcial de $I$ respecto de $x$, determinamos la rapidez con que la imagen cambia de intensidad a medida que $x$ cambia. Para funciones continuas, $I(x,y)$, podemos expresarlo de la siguiente manera:
	\begin{align}
		\frac{\partial I(x,y)}{\partial x} = \lim_{\nabla x\rightarrow 0} \frac{I(x + \nabla x, y) - I(x,y)}{\nabla x}	
	\end{align}
	
	 El cálculo de los gradientes de una imagen es útil ya que sirve, por ejemplo, para realizar detección de bordes de un objeto. La detección de bordes busca identificar puntos en una imagen en donde el brillo de la misma cambie de manera abrupta o, más formalmente, tenga discontinuidades. El propósito de esto es capturar eventos importantes o cambios en las propiedades de una imagen. En este caso, después de que los gradientes han sido computados, los píxeles con alto valor de gradiente son elegido como posibles bordes. Los píxeles con el valor de gradiente más alto en la dirección del gradiente se convierten en píxeles de borde. Los gradientes, también pueden ser usados en aplicaciones que realizan reconocimiento de objetos o correspondencia de texturas.	 