\section{Sistema Operativo, Hardware y Software}
	La mayor parte de los algoritmos fueron desarrollados en el Sistema Operativo \textit{Ubuntu (Linux)}. Al haber elegido Python como lenguaje de programación permite utilizarlos en distintas plataformas, como \textit{Mac} y \textit{Windows}, sin problemas.
	
	\subsection{Datos Específicos sobre Hardware Utilizado}
		\textbf{Computadora:}
		\begin{itemize}
			\item Procesador: 2.4 GHz Intel Core i7
			\item Memoria Ram: 16 GB 1333 Mhz DDR3
			\item Tipo de Arquitectura: i386-64bit
		\end{itemize}
		
	\subsection{Datos Específicos sobre Software Utilizado}
		\begin{itemize}
			\item Python 2.7.6 (Mar 22 2014, 22:59:56) 
			\item Pillow Version 2.4.0
			\item GCC 4.8.2
			\item numpy Version 1.8.1
			\item matplotlib Version 1.3.1
			\item ipython Version 2.1.0
			\item ipdb Version 0.8
			\item git Version 1.9.1
			\item scikit-image Version 0.10.0
			\item scipy Version 0.14.0
		\end{itemize}

    \newpage
	\subsection{Herramientas de Software utilizadas}
	    Durante la realización de la tesis, se usaron varias herramientas \textit{de software libre} que fueron indispensables y de una utilidad crucial. \\
	    \\
	   \textbf{ Algunas de estas herramientas fueron:}
	   \begin{itemize}

	       \item SUBLIME TEXT: IDE que se utilizó para escribir todo el código Python del presente trabajo\footnote{http://www.sublimetext.com/}.
	               	%--> Imagen ''sublime-text_logo.png''
					\begin{figure}[htbp]
						\centering
						\fbox{ \includegraphics[scale=0.35]{img/sublime_logo.png} }
						\caption{SUBLIME-TEXT IDE.}
						\label{fig:sublime_ide}
					\end{figure}

	       \item TexMaker: IDE que se usó para escribir todo el código \LaTeX{} de la tesis\footnote{http://www.xm1math.net/texmaker/}.
	               	%--> Imagen ''texmaker_logo.png''
					\begin{figure}[htbp]
						\centering
						\fbox{ \includegraphics[scale=1]{img/texmaker_logo.png} }
						\caption{Texmaker.}
						\label{fig:texmaker}
					\end{figure}

	       \item GIT: Sistema de control de versiones usado para todo el código\footnote{http://git-scm.com/}.
	               	%--> Imagen ''git_logo.png''
					\begin{figure}[htbp]
						\centering
						\fbox{ \includegraphics[scale=1]{img/git_logo.png} }
						\caption{GIT.}
						\label{fig:git}
					\end{figure}
	   \end{itemize}
	
	
	
	