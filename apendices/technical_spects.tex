\section{Sistema Operativo, Hardware y Software}
	La mayor parte de los algoritmos fueron desarrollados en el Sistema Operativo \textit{Ubuntu (Linux)}. Al haber elegido Python como lenguaje de programación permite utilizarlos en distintas plataformas, como \textit{Mac} y \textit{Windows}, sin problemas.
	
	\subsection{Datos Específicos sobre Hardware Utilizado}
		\textbf{Computadora\footnote{Utilizada solamente para procesar los resultados obtenidos del servidor y generar los gráficos necesarios.}:}
		\begin{itemize}
			\item Procesador: Intel\textregistered ~Core\texttrademark ~i7-4700MQ CPU @ 2.40GHz $\times$ 8 
			\item Memoria Ram: 16 GB 1333 Mhz DDR3
			\item Sistema Operativo: Ubuntu 14.04 LTS x64
		\end{itemize}
		
		\textbf{Servidor ``Ganesh''\footnote{Servidor proporcionado por la \textit{FaMAF} para correr los experimentos.}:}
		\begin{itemize}
			\item Micros: 4xDualCore AMD Opteron\texttrademark ~Processor 8212.
			\item Placa madre: SuperMicro H8QM8.
			\item Memoria Ram: 32 GB DDR2.
			\item Sistema Operativo: Ubuntu 12.04.5 LTS x64. 
		\end{itemize}
				
	\subsection{Datos Específicos sobre Software Utilizado}
		\begin{itemize}
			\item Python Versión 2.7.6
			\item Pillow Versión 2.4.0
			\item GCC Versión 4.8.2
			\item numpy Versión 1.8.1
			\item matplotlib Versión 1.3.1
			\item ipython Versión 2.1.0
			\item ipdb Versión 0.8
			\item git Versión 1.9.1
			\item scikit-image Versión 0.10.0
			\item scipy Versión 0.14.0
		\end{itemize}

	\subsection{Herramientas de Software utilizadas}
	    Durante la realización de la tesis, se usaron varias herramientas \textit{de software libre} que fueron indispensables y de una utilidad crucial. \\
	    \\
	   \textbf{ Algunas de estas herramientas fueron:}
	   \begin{itemize}

	       \item SUBLIME TEXT: IDE que se utilizó para escribir todo el código Python del presente trabajo\footnote{http://www.sublimetext.com/}.
	               	%--> Imagen ''sublime-text_logo.png''
					\begin{figure}[htbp]
						\centering
						\fbox{ \includegraphics[scale=0.35]{img/sublime_logo.png} }
						\caption{SUBLIME-TEXT IDE.}
						\label{fig:sublime_ide}
					\end{figure}

	       \item TexMaker: IDE que se usó para escribir todo el código \LaTeX{} de la tesis\footnote{http://www.xm1math.net/texmaker/}.
	               	%--> Imagen ''texmaker_logo.png''
					\begin{figure}[htbp]
						\centering
						\fbox{ \includegraphics[scale=1]{img/texmaker_logo.png} }
						\caption{Texmaker.}
						\label{fig:texmaker}
					\end{figure}

	       \item GIT: Sistema de control de versiones usado para todo el código\footnote{http://git-scm.com/}.
	               	%--> Imagen ''git_logo.png''
					\begin{figure}[htbp]
						\centering
						\fbox{ \includegraphics[scale=1]{img/git_logo.png} }
						\caption{GIT.}
						\label{fig:git}
					\end{figure}
	   \end{itemize}
	
	\subsection{Por qué Python?}	
	
	El lenguaje que se ha elegido para programar al clasificador como así también todos los módulos de procesamiento de resultados es Python. Las razones por las cuales se elige este lenguaje y no otro son muchas, entre las cuales algunas de las más importantes son:
	
			\begin{itemize}
	
			\item Python tiene una comunidad muy activa y cooperativa: La comunidad de programadores que programan con Python es muy grande (y tiende a seguir creciendo), y se caracteriza fuertemente por una muy buena aceptación de nuevos miembros y una gran predisposición a la ayuda mutua en medios como foros y listas de correos, como la lista de correo de PyAr (Python Argentina). Esto muestra que Python es un lenguaje vivo y en crecimiento.
			
			\item Python es un lenguaje con una sintaxis sencilla y es fácil de entender y aprender.
			
			\item Python es rápido: Python cuenta con una enorme cantidad de librerías magníficas implementadas en C (como NumPy o SciPy) que permiten realizar operaciones complejas en cuanto a procesamiento de manera sumamente rápida y eficiente. Incluso si no existiera una librería en particular en C para la funcionalidad que se desea, se puede implementar la porción de código que realiza la función crítica en C y utilizarla desde Python.
			
			\item Python tiene un gran soporte para computación científica: con librerías sólidas, fuertemente mantenidas, bien documentadas y áltamente eficientes, como SciPy, NumPy, matplotlib, entre otros.
			
			\item Python es multiplataforma: el código Python de este trabajo podrá ser utilizado prácticamente en cualquier sistema operativo. Esto es una ventaja importantísima y no limita al usuario a un sistema operativo o hardware en particular.
			\end{itemize}
	