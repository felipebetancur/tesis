\appendix
\section{Conceptos de probabilidad y notación}

	A continuación se introducirán algunos conceptos de probabilidad que serán de utilidad en el desarrollo de las secciones posteriores.

	\paragraph*{Variables aleatorias discretas} ~\\

		Dado un evento $A$, la expresión $p(A)$ denota la probabilidad de que $A$ sea verdadero. Por ejemplo, $A$ puede ser la expresión lógica ``Va a llover mañana''. Se requiere que $0 \leq p(A) \leq 1$, donde $p(A)=0$ significa que el evento no va a ocurrir, y $p(A)=1$ significa que el evento definitivamente va a suceder. Escribimos $p(\overline{A})$ para denotar la probabilidad del evento no $A$, es decir, de que el evento no ocurra; esto se define como $p(\overline{A})=1-p(A)$. Se escribirá $A=1$ para hacer referencia que el evento $A$ es verdadero, y $A=0$ cuando el evento $A$ es falso.
		
		Se puede extender la noción de eventos binarios definiendo una \textit{variable aleatoria discreta} $X$, la cual puede tomar cualquier valor de un conjunto finito o infinito contable \scalebox{1.4}{$\chi$}. Se denota la probabilidad de que $X=x$ por $p(X=x)$, o sólo $p(x)$ como abreviación. Ambas notaciones se van a usar indistintamente en el desarrollo del trabajo. Aquí $p: x \rightarrow \mathbb{R} $ se denomina \textit{función de probabilidad}. Esta satisface las propiedades $0 \leq p(x) \leq 1$ y $\sum_{x \in \chi}p(x)=1$.
		
	\paragraph*{Probabilidad de la unión de dos eventos} ~\\
		
		Dados dos eventos, $A$ y $B$, se define la probabilidad de A o B de la siguiente manera:
		\begin{align}
			p(A \lor B) &= p(A) + p(B) - p(A \land B) \\
			&= p(A) + p(B) ~\text{si A y B son mutuamente excluyentes}
		\end{align}
		
	\paragraph*{Probabilidad conjunta} ~\\
	
		Se define la probabilidad del evento conjunto A y B como sigue:
		\begin{align}
			p(A,B) = p(A \land B)
		\end{align}

		A esta ecuación se la llama \textit{regla del producto}. Dada la distribución conjunta en dos eventos p(A,B), se define la ``distribución marginal'' de la variable aleatoria $A$ de la siguiente manera:
		\begin{align}
			p(A)=\sum_{b}p(A,B=b)
		\end{align}
		donde se asumen todos los estados posibles de B. Se puede definir p(B) de forma similar. A esta ecuación se la denomina regla de la suma o regla de la probabilidad total.
		
	\paragraph*{Probabilidad condicional} ~\\
	
		Se define la probabilidad condicional del evento A, dado que el evento B es verdadero, como sigue:
		\begin{align}
			p(A|B) = \frac{p(A,B)}{p(B)} ~\text{si}~ p(B)>0
		\end{align}
		
	\paragraph*{Regla de Bayes} ~\\
		
		Combinando la definición de probabilidad condicional con las reglas del producto y la suma se obtiene la \textit{regla de Bayes}, también llamada \textit{Teorema de Bayes}:
		\begin{align}\label{eq:bayes}
			p(X=x|Y=y) &= \frac{p(X=x,Y=y)}{p(Y=y)} \\
			&= \frac{p(X=x)p(Y=y|X=x)}{\sum_{x'}p(X=x')p(Y=y|X=x')}
		\end{align}

	\paragraph*{Independencia e independencia condicional} ~\\
	
		Se dice que $X$ e $Y$ son independientes, denotado como $X \bot Y$	, si se puede representar la unión como el producto de dos marginales, es decir,
	\begin{align}
		X \bot Y \Longleftrightarrow p(X,Y) = p(X)p(Y)
	\end{align}			
			
		En general, se dice que un conjunto de variables es mutuamente independiente si la conjunta puede ser escrita como producto de marginales.
		
		Desafortunadamente, la independencia es rara, porque la mayoría de las variables pueden influir en la mayoría de las otras variables. Sin embargo, usualmente esta influencia se da a través de otras variables en vez de ser directa. Por lo tanto se dice que $X$ e $Y$ son \textit{condicionalmente independientes} dada $Z$ si y solo si la conjunta condicional puede ser escrita como producto de marginales condicionales:
		\begin{align}
			X \bot Y|Z \Longleftrightarrow p(X,Y|Z) = p(X|Z)p(Y|Z)
		\end{align}
	
